\section{Motivation}
\label{sec:motivation}
Das Ziel dieses Versuches ist die Bestimmung der \textsc{Debye}-Temperatur
$θ_D$ in Kupfer, dem verwendeten Material.
Dafür wird in der Theorie erstmal das klassische Modell zur Temperaturabhängigkeit
der Wärmekapazität $C(T)$ diskutiert und dann auf das \textsc{Debye}-Modell
eingegangen.

\section{Theoretische Grundlagen}
\label{sec:theorie}
Die Theorie folgt der Beschreibung in \cite{gross}.

Grundlegend ist die Wärmekapazität $C$ definiert als
\begin{align}
  C &= \frac{\Del{Q}}{\Del{T}}\,.
  \intertext{Damit nun unterschiedliche Materialien und Proben verglichen werden
    können, gibt es die molare Wärmekapazität}
  c^m &\equiv \frac{\Del{Q}}{\Del{T} \cdot \text{mol}}\,.\label{eqn:cp1}
  \intertext{Eine weitere Art ist die spezifische Wärmekapazität}
  c^{\text{mass}} &\equiv \frac{C}{m} \\
  c^{\text{vol}} &\equiv \frac{C}{V}\,.
  \intertext{Die \textsc{Debye}-Näherung wird interessant für $C_V$ und $C_p$.
    Diese folgen aus dem ersten Hauptsatz der Thermodynamik}
  \dif{Q} &= \dif{U} - \dif{W} = \dif{U} + p \dif{V}\,.
  \intertext{Die Indizes $V$ und $p$ geben an, welche Größe konstant ist}
  C_V &\equiv \left.\ableitung{Q}{T}\right|_V
    = \left.\ableitung{U}{T}\right|_V \\
  C_p &\equiv \left.\ableitung{Q}{T}\right|_p\,.
  \intertext{Ebenfalls aus der Thermodynamik folgt der Zusammenhang}
  C_p - C_V &= 9 α^2 κ V_0 T. \label{eqn:cp-cv}
\end{align}
Es treten auf die Temperatur $T$, das Volumen $V_0$, der Volumenausdehnungskoeffizient
$\alpha_V^2$ und die Kompressibilität $κ$.

\subsection{Klassische Betrachtung}
Wir wollen ein System aus $N$-Atomen betrachten, daraus folgen $3N$ Schwingungsmoden.
Durch den Gleichverteilungssatz wird der mittleren kinetischen und potentiellen
Energie jedes Atoms $\sfrac{1}{2} k_B T$ zugeordnet.
Die innere Energie ist somit
\begin{align}
  U &= U_{\text{Gleichgewicht}} + 3 N k_B T\,.
  \intertext{Damit ergibt sich}
  C_V &= 3 N k_B = 3 r' N' k_B
\end{align}
mit den neuen Konstanten der Anzahl an Einheitszellen $N'$ und $r'$ der
Anzahl an Atomen in einer Einheitszelle.
Dieses ist das \textsc{Dulong}-\textsc{Petit}-Gesetz.

\subsection{Einbeziehung der Quantenmechanik}
Werden die Schwingungen mit quantenmechanischen harmonischen Oszillatoren beschrieben,
folgt die Wärmekapazität
\begin{equation}
  C_V = \left.\ableitung{\!\left<U\right>}{T}\right|_V = \sum_{q,\,r}
    \ableitung{\,\!}{T} \frac{\hbarω_{q r}}
    {\exp\left(\frac{\hbarω_{q r}}{k_b T}\right) - 1}\,.
\end{equation}
Damit ergibt sich für hohe Temperaturen $k_B T ≫ \hbar ω_{q r}$
\begin{align}
  C_V &= 3 r' N k_B
  \intertext{und für tiefe Temperaturen $k_B T ≪ \hbar ω_{q r}$}
  C_V &= V \frac{2 \mpi^2}{5} k_B \left(\frac{k_B T}{\hbar v_s}\right)^{\!\!3}\,.
\end{align}

\subsection{\textsc{Debye}-Näherung}
Zum Start der \textsc{Debye}-Näherung werden folgende Annahmen getroffen:
\begin{enumerate}
  \item einatomige Basis: $r' = 1$
  \item Betrachtung der drei akustischen Phononenzweige mit Dispersion $ω_i = v_i q$
  \item Die Summation über $q$ wird zur Integration über die erste
    Brillouin-Zone. Dies kann in der Form einer Kugel mit Radius $q_D$ geschehen.
\end{enumerate}
Für den \textsc{Debye}-Wellenvektor $q_D$ gilt
\begin{equation}
  q_D = \sqrt[3]{6\mpi^2 \frac{N}{V}}\,.
\end{equation}
Die \textsc{Debye}-Temperatur $θ_D$ wird als
\begin{equation}
  θ_D \equiv \frac{\hbarω_D}{k_B} = \frac{\hbar v_s}{k_B} \sqrt[3]{6\mpi^2 \frac{N}{V}}
\end{equation}
formuliert. In den Hoch- und Tieftemperaturgrenzfällen gilt
\begin{equation}
  C_V^D =
  \begin{cases}
    \frac{12\mpi^4}{5}N k_B \left(\frac{T}{θ_D}\right)^{\!\!3} \qquad
      \text{für}~T ≪ θ_D \\
    3 N k_B \hspace{6.3em} \text{für}~T ≫ θ_D
  \end{cases}
\end{equation}

\subsection{Wärmetransport}
Der Wärmetransport kann auf drei Arten erfolgen:
\begin{itemize}
  \item Strahlung
  \item Konvektion
  \item Wärmeleitfähigkeit
\end{itemize}
Die Wärmeleitfähigkeit würde über die Probenhalterung erfolgen,
dies kann minimiert werden, indem Materialien verwendet die schlecht leiten.
Durch ein Vakuum wird der Wärmeverlust durch Konvektion minimiert.
Die Strahlungsverluste werden dadurch kompensiert, dass das Gehäuse indem die Probe ist,
auf der gleichen Temperatur gehalten wird.
