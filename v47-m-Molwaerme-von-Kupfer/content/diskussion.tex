\section{Diskussion}
\label{sec:Diskussion}
Während der Messung lagen die Temperaturen von Gehäuse und Probe in dem
vorgegebenen Differenzbereich, wie in Abbildung \ref{fig:temperatur} gut zu sehen ist.
Außerdem wurde im vorgegebenen Rahmen geheizt.

Die Auffälligkeiten der Wärmekapazitäten $C_P$ und $C_V$ passen mit dem Verlauf
der Temperaturkurve zusammen, so mussten teilweise, wie in den Messwerten zu sehen ist,
die Ströme angepasst werden, was in der Temperaturkurve zu sehen ist und dann auch in den
Plots von $C_P$ \ref{fig:cp1} und $C_V$ \ref{fig:cv1}.

Bei der Bestimmung der \textsc{Debye}-Temperatur $Θ_D$ ist der Fehler dadurch gegeben,
dass die bestimmten $C_V$ Werte teilweise zwischen den Einträgen in der
gegebenen Tabelle liegen und hier die Übertragung Ungenauigkeiten reinbringt.

Die \textsc{Debye}-Temperatur liegt nach \cite{debye-copper} im Bereich
$310 - \SI{347}{\kelvin}$. Das experimentelle Ergebnis liegt darunter, was an
der Tabelle liegen kann, das berechnete in dem Intervall.
