\section{Auswertung}
\label{sec:Auswertung}
\subsection{Temperaturverlauf}
In Abbildung \ref{fig:temperatur} ist der Temperaturverlauf während der
Messung zu sehen, die Werte von Probe und Gehäuse liegen nahe zusammen,
wie oben gefordert.
\begin{figure}
  \centering
  \includegraphics[width=0.8\textwidth]{build/p-temperatur.pdf}
  \caption{Temperaturverlauf der von Gehäuse und Probe während der Messung.}
  \label{fig:temperatur}
\end{figure}

\subsection{Wärmekapazitäten}
Die Wärmekapazität $C_P$ wird mit Gleichung \eqref{eqn:cp1} bestimmt.
Abgebildet sind die Werte in Abbildung \ref{fig:cp1}, die Werte stehen in Tabelle
\ref{tab:cp1}.
Die Einzelwerte sind
\begin{align}
  M &= \SI{63.546}{\gram\per\mol} &\text{\cite{kupfer-mol}}\\
  m &= \SI{342}{\gram} &\text{\cite{anleitung}}\\
  U_P &= \SI{17.64}{\volt}
\end{align}
\begin{figure}
  \centering
  \includegraphics[width=0.8\textwidth]{build/p-cp.pdf}
  \caption{Abhängigkeit der $C_P$-Werte von der Temperatur.}
  \label{fig:cp1}
\end{figure}

Für die Berechung von $C_V$ nach Gleichung \eqref{eqn:cp-cv} werden die Werte
\begin{align}
  κ &= \SI{140}{\giga\pascal} &\text{\cite{kupfer-kappa}} \\
  V_0 &= \frac{\text{molare Masse}}{\text{Dichte}} = \frac{63.546 \cdot N_A}{\SI{8.92}{\gram\per\cubic\centi\meter}}
\end{align}
verwendet. Für $α$ werden die Werte aus der Anleitung \cite{anleitung} an
\begin{align}
  α(T) &= a \cdot T^3 + b \cdot T^2 + c \cdot T + d
  \intertext{gefittet, es ergeben sich mit \texttt{scipy}}
  \input{build/p-alpha-a.tex} \\
  \input{build/p-alpha-b.tex} \\
  \input{build/p-alpha-c.tex} \\
  \input{build/p-alpha-d.tex}\,.
\end{align}
Das Ergebnis ist in Abbildung \ref{fig:alphafit} zu sehen.

\begin{figure}
  \centering
  \includegraphics[width=0.8\textwidth]{build/p-alphafit.pdf}
  \caption{Fit an die gegebenen Werte des linearen Ausdehnungskoeffizientens für Kupfer.}
  \label{fig:alphafit}
\end{figure}

Es folgen die $C_V$-Werte in Abbildung \ref{fig:cv1} und Tabelle \ref{tab:cv1}.

\begin{figure}
  \centering
  \includegraphics[width=0.8\textwidth]{build/p-cv.pdf}
  \caption{Abhängigkeit der $C_V$-Werte von der Temperatur.}
  \label{fig:cv1}
\end{figure}
\FloatBarrier
\subsection{Debye-Temperatur}
In Tabelle \ref{tab:debye1} stehen die $C_V$ mit den, nach der in der Anleitung
gegeben Tabelle, entsprechenden $\sfrac{θ_D}{T}$, sowie das Produkt dieser mit
der jeweiligen Temperatur.
Es ergibt sich eine mittlere \textsc{Debye}-Temperatur von
\begin{align}
  \input{build/p-debye.tex}\,.
\end{align}

\subsection{Betrachtung der Theorie}
Es soll
\begin{equation}
  \int_0^{ω_D} Z\l(ω\r)\,\dif{ω} \label{eqn:forderung}
\end{equation}
berechnet werden, um $ω_D$ und $θ_D$ zu bestimmen.
Gegeben sind die Schallgeschwindigkeiten
\begin{align}
  v_{\text{long}} &= \SI{4.7}{\kilo\meter\per\second} \\
  v_{\text{trans}} &= \SI{2.26}{\kilo\meter\per\second}\,.
\end{align}
Im \textsc{Debye}-Modell wird die spektrale Zustandsdichte $Z\l(ω\r)$ genähert,
mit der Forderung
\begin{itemize}
  \item Phasengeschwindigkeit nicht abhängig von Frequenz und Ausbreitungsrichtung.
\end{itemize}
So müssen nur die Eigenfrequenzen abgezählt werden und es folgt
\begin{align}
  Z\l(ω\r)\,\dif{ω} &= \frac{3 L^2}{2 \mpi^2 v^3} ω^2\, \dif{ω}\,
  \intertext{allerdings haben longitudinale und Transversale Wellen nicht
    unbedingt die gleiche Geschwindigkeit, sodass das Modell verbessert werden kann, zu}
  Z\l(ω\r)\,\dif{ω} &= \frac{L^2}{2 \mpi^2} ω^2 \left(\frac{1}{v_l^3} +
    \frac{1}{v_{tr}^3}\right)\,\dif{ω}\,.
  \intertext{Mit der Forderung \eqref{eqn:forderung} folgt}
  ω_D^3 &= \frac{18 \mpi^2 N_L}{L^3} \frac{1}{\frac{1}{v_l^3} + \frac{2}{v_{tr}^3}}
    = \SI{8.21e40}{\per\cubic\second}\\
  ω_D &= \SI{43.5e12}{\per\second} \\
  Z\l(ω\r)\,\dif{ω} &= \frac{9 N_L}{ω_D^3} ω^2\,\dif{ω}\,.
  \intertext{Setzt man dieses in die Integraldarstellung}
  C_V &= \frac{\ud}{\dif{T}} \int_0^{ω_D} Z\l(ω\r) \frac{\hbar ω}{\exp\left(
    \frac{\hbar ω}{k_B T}\right) - 1}
  \intertext{ein, folgt mit den Abkürzungen}
  x &:= \frac{\hbar ω}{k_B T} \\
  \frac{θ_D}{T} &:= \frac{\hbar ω_D}{k_B T} \\
  C_{V,\textsc{Debye}} &= 9 R \left(\frac{θ_D}{T}\right)^{\!\!3} \int_0^{\frac{θ_D}{T}}
    \frac{x^4 \me^x}{\left(\me^x - 1\right)^2}\,.
  \intertext{Aus der Integrationsgrenze können wir}
  θ_D &= \frac{\hbar ω_D}{k_B} = \SI{331.96}{\kelvin}
\end{align}
bestimmen.
