\documentclass[
  bibliography=totoc,     % Literatur im Inhaltsverzeichnis
  captions=tableheading,  % Tabellenüberschriften
  titlepage=firstiscover, % Titelseite ist Deckblatt
]{scrartcl}

% Paket float verbessern
\usepackage{scrhack}

% Warnung, falls nochmal kompiliert werden muss
\usepackage[aux]{rerunfilecheck}

% unverzichtbare Mathe-Befehle
\usepackage{amsmath}
% viele Mathe-Symbole
\usepackage{amssymb}
% Erweiterungen für amsmath
\usepackage{mathtools}

% Fonteinstellungen
\usepackage{fontspec}
% Latin Modern Fonts werden automatisch geladen
% Alternativ zum Beispiel:
%\setromanfont{Libertinus Serif}
%\setsansfont{Libertinus Sans}
%\setmonofont{Libertinus Mono}

% Wenn man andere Schriftarten gesetzt hat,
% sollte man das Seiten-Layout neu berechnen lassen
\recalctypearea{}

% deutsche Spracheinstellungen
\usepackage[main=ngerman]{babel}


\usepackage[
  math-style=ISO,    % ┐
  bold-style=ISO,    % │
  sans-style=italic, % │ ISO-Standard folgen
  nabla=upright,     % │
  partial=upright,   % ┘
  warnings-off={           % ┐
    mathtools-colon,       % │ unnötige Warnungen ausschalten
    mathtools-overbracket, % │
  },                       % ┘
]{unicode-math}

% traditionelle Fonts für Mathematik
\setmathfont{Latin Modern Math}
% Alternativ zum Beispiel:
%\setmathfont{Libertinus Math}

\setmathfont{XITS Math}[range={scr, bfscr}]
\setmathfont{XITS Math}[range={cal, bfcal}, StylisticSet=1]

% Zahlen und Einheiten
\usepackage[
  locale=DE,                   % deutsche Einstellungen
  separate-uncertainty=true,   % immer Fehler mit \pm
  per-mode=symbol-or-fraction, % / in inline math, fraction in display math
]{siunitx}
\DeclareSIUnit{\year}{a}
\sisetup{math-micro=\text{µ},text-micro=µ}


% chemische Formeln
\usepackage[
  version=4,
  math-greek=default, % ┐ mit unicode-math zusammenarbeiten
  text-greek=default, % ┘
]{mhchem}

% richtige Anführungszeichen
\usepackage[autostyle]{csquotes}

% schöne Brüche im Text
\usepackage{xfrac}

% Standardplatzierung für Floats einstellen
\usepackage{float}
\floatplacement{figure}{htbp}
\floatplacement{table}{htbp}
\usepackage{pgfplotstable}
\usepackage{array}

% Floats innerhalb einer Section halten
\usepackage[
  section, % Floats innerhalb der Section halten
  below,   % unterhalb der Section aber auf der selben Seite ist ok
]{placeins}

% Seite drehen für breite Tabellen: landscape Umgebung
\usepackage{pdflscape}

% Captions schöner machen.
\usepackage[
  labelfont=bf,        % Tabelle x: Abbildung y: ist jetzt fett
  font=small,          % Schrift etwas kleiner als Dokument
  width=0.9\textwidth, % maximale Breite einer Caption schmaler
]{caption}
% subfigure, subtable, subref
\usepackage{subcaption}

% Grafiken können eingebunden werden
\usepackage{graphicx}

% PDF's können als weitere Seite eingefügt werden
\usepackage{pdfpages}

% Grafiken im Text einbinden
\usepackage{wrapfig}

% schöne Tabellen
\usepackage{booktabs}

% Verbesserungen am Schriftbild
\usepackage{microtype}

% Tikzpictures mitverwenden
\usepackage{tikz}
\usepackage{circuitikz}

%Matrizen mit doppeltem Unterstrich einbinden
\usepackage{ulem}

%Neue Commands für erweiterte Funktionen
\newcommand \matrize[1]{\underline{\underline{#1}}}
\newcommand \laplace{\symup{Δ}}
\newcommand \uD{\symup{Δ}}
\newcommand \abs[1]{\left|#1\right|}
\newcommand \glname[2]{\textsc{#1}~#2}
\newcommand \ud{\symup{d}}
\newcommand \alphat{$α$-Teilchen}
\newcommand \ableitung[2]{\frac{\partial #1}{\partial #2}}

\usepackage{mleftright}
\DeclarePairedDelimiter{\bra}{\langle}{\rvert}
\DeclarePairedDelimiter{\ket}{\lvert}{\rangle}
\DeclarePairedDelimiterX{\braket}[2]{\langle}{\rangle}{#1 \delimsize| #2}

\usepackage{expl3}
\usepackage{xparse}
\ExplSyntaxOn
\NewDocumentCommand \I {} {\symup{i}}
\NewDocumentCommand \me {} {\symup{e}}
\NewDocumentCommand \mpi {} {\symup{π}}
\NewDocumentCommand \grau {m} {\textcolor{gray}{#1}}
\NewDocumentCommand \rot {m} {\textcolor{red}{#1}}
\NewDocumentCommand \tug {m} {\textcolor{tugreen}{#1}}
\NewDocumentCommand \zB {} {z.\,B.~}
\NewDocumentCommand \DaH {} {d.\,h.~}
\NewDocumentCommand \zZ {} {z\!Z}
\NewDocumentCommand \dif {m} {\mathinner{\mathrm{d} #1}}
\NewDocumentCommand \del {m} {\mathinner{\mathrm{δ} #1}}
\NewDocumentCommand \Del {m} {\mathinner{\mathrm{Δ} #1}}
\NewDocumentCommand \IN {} {^{-1}}
\NewDocumentCommand \bfE {} {\symbf{E}}
\let\ltext=\l
\RenewDocumentCommand \l {} {\TextOrMath{ \ltext }{ \mleft }}
\let\raccent=\r
\RenewDocumentCommand \r {} {\TextOrMath{ \raccent }{ \mright }}
\ExplSyntaxOff
% Literaturverzeichnis
\usepackage[
  backend=biber,
]{biblatex}
% Quellendatenbank
\addbibresource{lit.bib}
\addbibresource{programme.bib}

% Hyperlinks im Dokument
\usepackage[
  unicode,        % Unicode in PDF-Attributen erlauben
  pdfusetitle,    % Titel, Autoren und Datum als PDF-Attribute
  pdfcreator={},  % ┐ PDF-Attribute säubern
  pdfproducer={}, % ┘
]{hyperref}
% erweiterte Bookmarks im PDF
\usepackage{bookmark}

% Trennung von Wörtern mit Strichen
\usepackage[shortcuts]{extdash}

% Einrückung zu Beginn des Paragraphen ausstellen
\setlength\parindent{0pt}

%Benutzung von Schleifen
\usepackage{ifthen}
\usepackage{forloop}

%Fuer Tabellen oder CSV allg
\usepackage{csvsimple}

% Aufteilen von Seiten
\usepackage{multicol}
\usepackage[paper=a4paper, bottom=50mm, top=50mm]{geometry}

\author{%
  Schokoladenporsche
}

\publishers{TU Dortmund – Fakultät Physik}
