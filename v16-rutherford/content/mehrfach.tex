\subsection{Mehrfachstreuungen}
In Abbildung \ref{fig:mehrfach} sind die Counts für verschiedene
Foliendicken bei einem Winkel von $\SI{0}{\degree}$ gegen die Foliendicke aufgetragen.
Die Werte stehen in Tabelle \ref{tab:mehrfach}.

\begin{table}
    \centering
    \caption{Messwerte für verschiedene Foliendicken.}
    \label{tab:mehrfach}
    \begin{tabular}{S[table-format=1.0] S[table-format=3.0]
        S[table-format=4.0] @{${}\pm{}$} S[table-format=2.0]
        S[table-format=2.2] @{${}\pm{}$} S[table-format=1.1]
        S[table-format=2.1] @{${}\pm{}$} S[table-format=1.1]}
        \toprule
        {$x\:/\:\si{\micro\meter}$} &
        {$t_\text{int}\:/\:\si{\second}$} &
        \multicolumn{2}{c}{Counts} &
        \multicolumn{2}{c}{$\frac{\text{Counts}}{\si{\second}}$} &
        \multicolumn{2}{c}{$\frac{\symup{d\sigma}}{\symup{d\Omega}}\,[\si{\per\meter\squared}]$} \\
        \midrule
        0 & 100 & 1610 & 40 & 16.1  & 0.4 &       &     \\
        2 & 100 &  924 & 30 &  9.24 & 0.3 & 85    & 3   \\
        4 & 240 & 1271 & 36 &  5.3  & 0.2 & 24.2  & 0.9 \\
        \bottomrule
    \end{tabular}
\end{table}

\begin{figure}
    \centering
    \includegraphics[width=0.8\textwidth]{build/mehrfach.pdf}
    \caption{Plots der Mehrfachstreuungen.}
    \label{fig:mehrfach}
\end{figure}
