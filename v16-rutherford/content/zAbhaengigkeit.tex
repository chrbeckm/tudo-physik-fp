\subsection{Z-Abhängigkeit der Streumaterialien bei einem großen Winkel}
Die Z-Abhängigkeit der Streumaterialien wurde bestimmt, indem
die Ordnungszahl $Z$ gegen den Quotienten ${\frac{I}{N\cdot x}}$
bei einem Winkel von $\SI{20}{\degree}$ aufgetragen wird.
Dabei ist $I$ die Intensität der $\alpha$-Teilchen in der Einheit
\begin{equation}
  \big[ I \big] = \Big[ \frac{\text{Counts}}{\text{Zeiteinheit}} \Big]\,,
\end{equation}
N ist die Anzahl der Streuzentren, welche sich aus der Teilchendichte des Materials bestimmt, und x ist die Dicke der Streufolie.
Die verwendeten Parameter sind in Tabelle \ref{tab:params}  aufgetragen.

\begin{table}
  \centering
  \caption{Parameter für die Messung der Z-Abhängigkeit.}
  \label{tab:params}
  \begin{tabular}{c| c c c}
    \hline
    & \multicolumn{3}{c}{Material} \\
    \hline
    \text{Parameter} & Al & Au & Bi \\
    \hline
    $\rho \, [\si{\gram\per\centi\meter\tothe{3}}]$ & 2.7 & 19.32 & 9.80 \\
    $M_{mol} \, [\si{\gram\per\mole}]$ & 27 & 197 & 209 \\
    $N \, \si{\per\cubic\meter}$ & $\num{6.022e28}$ & $\num{5.906e28}$ & $\num{2.824e28}$ \\
%   $\rho_{N} \, [\si{\per\centi\metre\tothe{3}}]$ & $\num{6.022e18}$ & $\num{5.906e18}$ & $\num{2.823e18}$ \\
    $x \, [\si{\micro\meter}]$ & 3 & 4 & 2 \\
    $Z$ & 13 & 79 & 83 \\
    $I \, [\si{\per\second}]$ & $\num{0.023(9)}$ & $\num{0.20(3)}$ & $\num{0.027(9)}$ \\
    $\text{Counts}$ & 7 & 61 & 8 \\
    $\symup{t} [\si{\second}]$ & 300 & 300 & 300 \\
    $\sfrac{I}{N\cdot x\,[\SI{e-25}{\meter\squared\per\second}]}$ &  $\num{1.3(5)}$ & $\num{9(1)}$ & $\num{5(2)}$ \\
  \end{tabular}
\end{table}

Der Fehler auf die Intensität ist aus der Poissonverteilung kommend
\begin{equation}
  \Delta I = \sqrt{I}\,,
\end{equation}
erst danach wird durch die Integrationszeit geteilt.
$\symup{\Delta t}$ ist die Integrationszeit.
Der mit diesen Daten entstandene Plot ist in Abbildung \ref{fig:zab} abgebildet.

\begin{figure}
  \centering
  \includegraphics{build/z_abh.pdf}
  \caption{Z-Abhängigkeit der Streumaterialien.}
  \label{fig:zab}
\end{figure}
