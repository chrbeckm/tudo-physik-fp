\section{Diskussion}
\label{sec:Diskussion}
Die erste Messung zeigt deutlich den erwarteten Einfluss eines vorgeschalteten Verstärkers.
Ohne Verstärker sieht man eine instantane Änderung des Signals, welches dann abklingt.
Dies liegt am Detektor in dem Nachentladungen entstehen können.
Bei einem vorgeschaltetem Verstärker steigt das Signal sichtbar langsamer an, es ist also nicht sehr scharf.
Dies liegt an den elektronischen Komponenten, vor allem den Spulen im Gerät.
Die Impulshöhe ohne Verstärker ist sichtbar kleiner als die mit Verstärker.

Die Messung des Energieverlustes führt auf eine Foliendicke von
$\SI{3.5(7)}{\micro\meter}$, was in der Größenordnung der tatsächlichen Dichte von
$\SI{2}{\micro\meter}$ liegt, aber knapp außerhalb der doppelten Fehlerreichweite.
Mit Messungen bei weiteren Winkeln und längeren Integrationszeiten

Die Messung kann mit einer längeren Integrationszeit
und einer feineren Winkelauflösung verbessert werden.
Der Wert bei $φ = \SI{17.1}{\degree}$ passt nicht zu den anderen.
Hier liegt eine unbekannte externe Störung vor,
da das Phänomen in einigen Probemessungen auch beobachtet werden konnte.
Auch die Berechnung der Flugstrecke der $\alpha$-Teilchen in Luft liefert ein
vernünftiges Ergebnis.

Die Messung der Mehrfachstreuungen an verschieden dicken Goldfolien zeigt, dass
die Intensität mit der Dicke der Goldfolien abnimmt, was die Theorie einer
Mehrfachstreuung an den verschiedenen Schichten der Folie bestätigt.

Bezüglich der Z-Abhängigkeit der Streumaterialien ist keine Aussage zu
machen. Zu erwarten ist, das mit größerem Z die Intensität kleiner wird,
da es wesentlich mehr Streuzentren gibt, an welchen die $\alpha$-Teilchen
ihre Energie abgeben können.
Wir haben den "großen" Winkel von $\SI{20}{\degree}$ gewählt, da bei noch
größeren Winkel die Intensität ziemlich genau Null war. Wir haben
testweise für $\SI{300}{\second}$ gemessen um zu schauen wie viele Counts wir
erhalten und haben schnell festgestellt, dass wir mit dieser Zeit nur sehr
wenige Counts erhalten und die Messzeit um einige Größenordnungen höher
sein müsste um die Unsicherheit gering zu halten. Demnach ist unsere Messung
nicht sehr aussagekräftig, aufgrund der kleinen Statistik.
