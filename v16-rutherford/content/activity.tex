\subsection{Aktivität der Probe}
Die theoretische Aktivität der Probe beträgt
\begin{equation}
    A = \SI{330}{\kilo\becquerel}\cdot\exp\left(-\frac{\ln(2)}{\SI{423}{\year}}\SI{24.6}{\year}\right)
    = \SI{317.23}{\kilo\becquerel}\,.
\end{equation}
Für die experimentelle Aktivität wird zuerst die effektive Detektorfläche bestimmt.
Vor dem Detektor ist im Abstand von $\SI{4}{\milli\meter}$ eine
$\SI{2}{\milli\meter}\times\SI{10}{\milli\meter}$ Blende montiert.
Mit einem Strahlensatz kann das Verhältnis von Folie-Blende zu Folie-Detektor bestimmt werden.
Es gilt nach Abbildung~\ref{fig:aufbau}
\begin{align}
    \frac{\SI{97}{\milli\meter}}{\SI{101}{\milli\meter}} &= \frac{\text{Blende}}{\text{Detektor}}
\end{align}
Es folgen
\begin{align*}
    \text{Detektor}_\text{Breite} &= \SI{106.46341}{\milli\meter} \\
    \text{Detektor}_\text{Höhe} &= \SI{110.85365}{\milli\meter}\,.
\end{align*}
Der gesuchte Raumwinkel ist
\begin{equation}
    \laplaceΩ = 4\cdot\arctan\left(\frac{B\cdot H} {2\cdot\SI{101}{\milli\meter}\cdot\sqrt{4\cdot\SI{45}{\milli\meter}+B^2+H^2}}\right)
    = \num{0.90495}\,.
\end{equation}
Die Aktivität folgt mit
\begin{equation}
    A_\text{exp} = I_0\frac{4\mpi}{\laplaceΩ} = \SI{224(6)}{\kilo\becquerel}\,.
\end{equation}
