\subsection{Energieverlust}
In Abbildung \ref{fig:energieverlust} bzw. \ref{fig:energieverlust_lin} sind
die Daten der Energieverlustmessung
aufgetragen. Es wurde zwischen der maximalen und minimalen Amplitude
nach
\begin{align}
	\overline{x} &= \frac{1}{N} \sum_{i=0}^{N} x_i
	\label{eqn:mittelwert}
  \shortintertext{der Mittelwert und nach}
  \increment\overline{x} &= \sqrt{
	\frac{1}{N(N-1)}\sum_{k=0}^{N}
	\left( x_k - \overline{x} \right)^2}
	\label{eqn:mittelwertfehler}
	\intertext{der Fehler des Mittelwerts bestimmt.
		Alle Daten stehen in den Tabellen
		\ref{tab:energieverlust-ohne-folie} \&
		\ref{tab:energieverlust-mit-folie}.
		Die Ausgleichsgeraden der Form}
	U &= m\cdot p+b
	\intertext{führen auf die Parameter}
	\input{build/energieverlust-ohne-folie-m.tex} \\
	\input{build/energieverlust-ohne-folie-b.tex}
	\intertext{für die Messung ohne Folie. Mit der $\SI{2}{\micro\meter}$ Goldfolie folgen}
	\input{build/energieverlust-mit-folie-m.tex} \\
	\input{build/energieverlust-mit-folie-b.tex}\,.
	\intertext{Die Energiedifferenz berechnet sich gemäß}
	\increment E &= \symup{E}_{\alpha}\cdot\left(1\,-\,\frac{b_{mit}}{b_{ohne}}\right)\,.
	\shortintertext{Damit folgt}
	\input{build/energieverlust-delta-e.tex}\:. \\
	E_α &= \SI{5485.56}{\kilo\electronvolt}
	\intertext{Die Geschwindigkeit der \alphat~folgt mit umstellen der kinetischen Energie}
	E_\text{kin} &= \frac{mv^2}{2}
	\shortintertext{zu}
	v_α^2 &= \frac{E_\text{ohne} + E_\text{mit}}{m_α}
	\shortintertext{mit \cite{consts}}
	m_α &= \SI{3727.379378}{\mega\electronvolt} \\
	v_α &= \sqrt{
		\frac{E_α}{m_α}
		\left( 1\,+\,\frac{b_\text{mit}}{b_\text{ohne}}\right)} \\
	\input{build/energieverlust-v.tex} \\
	\input{build/energieverlust-c.tex} \,.
	\intertext{Mit der nach $\laplace x$ umgestellten
		\glname{Bethe-Bloch}{-Formel}
		\eqref{eqn:bethe-bloch} ergibt sich}
	\laplace x &= \frac{ \laplace E \cdot m_\text{e} \cdot v_α^2 \cdot 4 \mpi \cdot ε_0^2 \cdot A \cdot u }
	{ z^2 \cdot \symup{e}^4 \cdot Z \cdot ρ \cdot \ln \left(\frac{2 m_\text{e} v_α^2}{I}\right)} \\
	\input{build/energieverlust-deltax.tex}\,.
	\intertext{Der Energieverlust ist}
	\frac{\ud E}{\ud x} &= \frac{\laplace E}{\laplace x}
		= \SI{1.12(29)e12}{\electronvolt\per\meter}\,.
\end{align}

\begin{align*}
	\intertext{Die Konstanten sind\cite{consts}\cite{Eohne}:}
	\input{build/energieverlust-konstanten.tex}\:.
\end{align*}

Analog wurde auch die Reichweite von $α$-Strahlung in Luft berechnet,
wobei anstatt mit Luft mit Stickstoff gerechnet wurde, da diese zu einem
großen Prozentsatz daraus besteht.
Die Reichweite in Luft berechnet sich zu
\begin{align}
	\laplace x &= \SI{1.6(3)}{\centi\metre}\,,
	\intertext{der Energieverlust zu}
	\frac{\ud E}{\ud x} &= \frac{\laplace E}{\laplace x} = \SI{2.4(6)e8}{\electronvolt\per\meter}\,.
\end{align}
Die neuen Parameter dafür sind:\\
\begin{align*}
A &= \num{14}\\
z &= \num{2}\\
Z &= \num{7}\\
v_{\alpha} &= \SI{1.626e7}{\metre\per\second}\\
\rho &= \SI{1.2041}{\kilo\gram\per\metre\tothe{3}}\\
I &= \SI{1.121e-17}{\joule}\\
\end{align*}
Mit einer anderen Variante der \glname{Bethe-Bloch}{-Gleichung}
\begin{align}
	\frac{\laplace E}{\laplace x} &= \frac{4\mpi \symup{e}^2z^2N^2}{m_0v^2(4\mpiε_0)^2}\ln\left(\frac{2m_0v^2}{I}\right) \\
	N^2 &= \frac{m_0v^2(4\mpiε_0)^2}{4\mpi \symup{e}^2z^2\ln\left(\frac{2m_0v^2}{I}\right)}\frac{\laplace E}{\laplace x} \\
	N^2 &= \frac{ \SI{6.6e-27}{\kilo\gram}\cdot v_α^2 (4\mpiε_0)^2 }
			{4\mpi (\SI{1.602e-19}{\coulomb})^2\cdot 4\cdot
			\ln\left( \frac{2\cdot\SI{6.6e-27}{\kilo\gram} v_α^2 } {\SI{1.21e-17}{\joule}} \right)} \frac{\SI{5.485}{\mega\electronvolt}}{\SI{10.1}{\centi\meter}} \\
	N &= \SI{9.99e-5}{\per\cubic\meter} \\
	\intertext{folgt der Druck}
	p &= Nk_\text{b}T = \SI{4.3e-25}{\pascal}\,,
\end{align}
bei dem kein Teilchen mehr am Detektor gemessen wird, bei einer Temperatur $T=\SI{310}{\kelvin}$.
\begin{figure}
	\centering
	\includegraphics[width=0.8\textwidth]{build/energieverlust.pdf}
    \caption{Messwerte der Energieverlustmessung und die zugehörigen Ausgleichsgeraden, logarithmisch.}
    \label{fig:energieverlust}
\end{figure}

\begin{figure}
	\centering
	\includegraphics[width=0.8\textwidth]{build/energieverlust_linear.pdf}
    \caption{Messwerte der Energieverlustmessung und die zugehörigen Ausgleichsgeraden.}
    \label{fig:energieverlust_lin}
\end{figure}


\begin{table}
  \centering
  \begin{minipage}{.5\linewidth}
    \centering
    \caption{Ohne Folie.}
    \label{tab:energieverlust-ohne-folie}
    \sisetup{table-format=1.2}
    \begin{tabular}{S[table-format=3.2] S S S @{${}\pm{}$} S}
      \toprule
      {$p\:/\:\si{\milli\bar}$} &
      {$U_\text{max}\:/\:\si{\volt}$} &
      {$U_\text{min}\:/\:\si{\volt}$} &
      \multicolumn{2}{c}{$\bar{U}\:/\:\si{\volt}$} \\
      \midrule
        0.06 & 4.6  & 3.64 & 4.12 & 0.48 \\
        6.0  & 4.6  & 3.56 & 4.08 & 0.52 \\
       10.0  & 4.6  & 3.56 & 4.08 & 0.52 \\
       20.1  & 4.56 & 3.36 & 3.96 & 0.60 \\
       29.4  & 4.48 & 3.24 & 3.86 & 0.62 \\
       39.3  & 4.32 & 3.24 & 3.78 & 0.54 \\
       51.4  & 4.16 & 3.12 & 3.64 & 0.52 \\
       97.9  & 3.64 & 2.60 & 3.12 & 0.52 \\
      148.4  & 2.92 & 1.66 & 2.29 & 0.63 \\
      198.7  & 2.22 & 1.02 & 1.62 & 0.60 \\
      \bottomrule
    \end{tabular}
  \end{minipage}
  
  \begin{minipage}{.5\linewidth}
    \centering
    \caption{Mit $\SI{2}{\micro\meter}$-Goldfolie.}
    \label{tab:energieverlust-mit-folie}
    \sisetup{table-format=1.3}
    \begin{tabular}{S[table-format=3.2] S S S @{${}\pm{}$} S}
      \toprule
      {$p\:/\:\si{\milli\bar}$} &
      {$U_\text{max}\:/\:\si{\volt}$} &
      {$U_\text{min}\:/\:\si{\volt}$} &
      \multicolumn{2}{c}{$\bar{U}\:/\:\si{\volt}$} \\
      \midrule
        0.04 & 3.82  & 2.48  & 3.15  & 0.67  \\
        0.11 & 3.82  & 2.36  & 3.09  & 0.73  \\
        0.2  & 3.84  & 2.42  & 3.13  & 0.71  \\
        0.32 & 3.84  & 2.26  & 3.05  & 0.79  \\
        0.39 & 3.82  & 2.25  & 3.035 & 0.785 \\
        0.5  & 3.72  & 2.38  & 3.05  & 0.67  \\
        0.65 & 3.70  & 2.32  & 3.01  & 0.69  \\
        0.79 & 3.78  & 2.20  & 2.99  & 0.79  \\
        0.99 & 3.74  & 2.10  & 2.92  & 0.82  \\
        2.1  & 3.66  & 2.14  & 2.9   & 0.76  \\
        3.2  & 3.68  & 2.30  & 2.99  & 0.69  \\
        4.4  & 3.70  & 2.20  & 2.95  & 0.75  \\
        5.3  & 3.58  & 2.32  & 2.95  & 0.63  \\
        6.2  & 3.62  & 2.14  & 2.88  & 0.74  \\
        7.1  & 3.74  & 2.20  & 2.97  & 0.77  \\
        8.3  & 3.60  & 2.10  & 2.85  & 0.75  \\
        9.2  & 3.56  & 2.04  & 2.8   & 0.76  \\
       10.0  & 3.42  & 2.14  & 2.78  & 0.64  \\
       21.8  & 3.40  & 2.14  & 2.77  & 0.63  \\
       30.1  & 3.42  & 2.08  & 2.75  & 0.67  \\
       40.0  & 3.12  & 1.88  & 2.5   & 0.62  \\
       49.9  & 3.28  & 1.96  & 2.62  & 0.66  \\
       99.0  & 2.52  & 1.08  & 1.8   & 0.72  \\
      148.7  & 1.66  & 0.86  & 1.26  & 0.4   \\
      197.7  & 0.936 & 0.712 & 0.824 & 0.112 \\
      \bottomrule
    \end{tabular}
  \end{minipage}
\end{table}

