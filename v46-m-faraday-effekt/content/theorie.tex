\section{Motivation}
\label{sec:motivation}
Das Ziel dieses Versuches ist die Bestimmung der effektiven Masse von
Elektronen in Halbleitern.
In diesem Fall sind die Halbleiter n-dotierte GaAs Kristalle.
Für diese Bestimmung wird der \textsc{Faraday}-Effekt genutzt.

\section{Theoretische Grundlagen}
\label{sec:theorie}
\subsection{Dotierung von Halbleitern}
In einem GaAs Halbleiter sind die je drei Valenzelektronen der Gallium
und je fünf der Arsenatome in den Bindungen zwischen den Atomen.

Die Dotierung des Halbleiters kann mit zwei Methoden passieren.
In der einen wird der Halbleiter erwärmt und in das Gas des Fremdatoms gesetzt,
dann können diese in den Kristall diffundieren.
Die zweite Methode beruht auf dem Einschuss der Fremdatome.
Die Dotierung beträgt meist zwischen $10^{-4}$ und $10^{-8}$ \cite{demtroeder}.

Wenn mit einem Atom aus der fünften Hauptgruppe dotiert wird,
ist ein Elektron nicht gebunden, aufgrund der geringen Anzahl an Fremdatomen
kann dieses als einzelnes \enquote{freies} Elektron gesehen werden.

Die zusätzliche positive Ladung des Atomrumpfes wird größtenteils von den
Elektronen in den Bindungen kompensiert, sodass die Coulombkraft nicht mehr
komplett auf das Elektron wirkt.

\subsection{Effektive Masse von Elektronen in Halbleiterstrukturen}
Wie in Abschnitt \ref{sec:1.4} zu sehen ist,
wird im Kristall ein äußeres elektrisches Feld angelegt.
Durch die Kristallstruktur wirkt auf das Elektron auch ein ortsabhängiges Potential.
Damit trotzdem wichtige Formeln weiterhin ohne Korrekturen gelten,
wie z.\,B. das ohmsche Gesetz
\begin{align*}
  \vec{j} &= \sigma \vec{E}\,,
  \intertext{wird eine effektive Masse $m^\ast$  für die Elektronen definiert.
    Die Herleitung \cite{demtroeder} startet bei der Gruppengeschwindigkeit}
  v_g &= \frac{1}{\hbar}\frac{\dif{E}}{\dif{k}}
  \shortintertext{und kann mit der zeitlichen Ableitung und}
  F &= m^\ast \cdot a = \hbar \frac{\dif{k}}{\dif{t}}
  \shortintertext{umgeformt werden zu}
  m^\ast &= \hbar^2 \cdot \left(\frac{\dif{^2E}}{\dif{k_i}\dif{k_j}}\right)^{-1}\,.
\end{align*}
Allgemein ist diese Größe richtungsabhängig, im vorliegenden Material GaAs
aufgrund der Symmetrie des Kristalls nicht.

\subsection{Doppelbrechende Kristalle}
Fällt ein zirkular polarisierter Lichtstrahl auf einen doppelbrechenden
Kristall, kommen zwei orthogonal linearpolarisierte Strahlen auf der anderen
Seite heraus.
Dies liegt an den unterschiedlichen Brechungsindizes für die beiden
Polarisationsrichtungen.

Einige Kristalle sind immer doppelbrechend.
Andere können mit einem Magnetfeld angeregt werden.

\subsection{\textsc{Faraday}-Effekt}
\label{sec:1.4}
Die durch die Probe, den Halbleiter, gehenden Lichtstrahlen können als
zirkular polarisierte elektrische Felder aufgefasst werden.
Diese Felder üben die Kraft
\begin{equation*}
  \vec{F}_E = - \me_0 \cdot \vec{E}
\end{equation*}
auf die Elektronen aus.
Da bewegte elektrische Ladungen Magnetfelder induzieren,
wird das außen angelegte Magnetfeld entsprechend beeinflußt.
Damit das induzierte Magnetfeld mit dem angelegten Magnetfeld zusammen
die Doppelbrechung hervorruft, müssen Magnetfeld und einfallender Lichtstrahl parallel liegen.

Das Magnetfeld, insgesamt, kann dafür verwendet werden, die Polarisationsebenen
zu drehen, um den Winkel
\begin{equation*}
  α = V \cdot B \cdot l\,,
\end{equation*}
wobei $B$ das Magnetfeld, $l$ die Länge des Kristalls und $V$ die
materialspezifische \enquote{Verdet-Konstante} ist, welche nicht konstant,
sondern von unter anderem der Wellenlänge abhängig ist \cite{heintze}.

\subsection{\textsc{Faraday}-Rotation}
Die mathematische Beschreibung des Problems startet mit einer Welle die sich
in $z$-Richtung ausbreitet
\begin{align*}
  E\l(z\r) &= \frac{1}{2} \left(E_R\l(z\r) + E_L\l(z\r)\right)\,,
  \intertext{wobei $E_R$ und $E_L$ rechts- und linkszirkular polarisierte Wellen sind}
  E_R\l(z\r) &= \left(E_0 \vec{x}_0 - \I E_0 \vec{y}_0\right) \me^{\I k_R z} \\
  E_L\l(z\r) &= \left(E_0 \vec{x}_0 + \I E_0 \vec{y}_0\right) \me^{\I k_L z}\,.
\end{align*}
Nach dem Einsetzen dieser Wellenfunktionen und Betrachtung des Systems
am Ende des Kristalls der Länge $L$ können die Abkürzungen
\begin{align}
  ψ &:= \frac{L}{2} \l(k_R + k_L\l) \\
  θ &:= \frac{L}{2} \l(k_R - k_L\l) \label{eqn:theorietheta}
\end{align}
eingeführt werden und es folgt
\begin{equation*}
  E\l(L\r) = E_0 \me^{i ψ} \left(\cos\l(θ\vec{x}_0\r) + \I \sin\l(υ\vec{y}_0\r)\right)\,.
\end{equation*}
Dies beschreibt eine Welle, die um $θ$ zur Einfallsebene gedreht ist.
Mit den Beziehungen
\begin{align*}
  v_{\text{ph}} &= \frac{ω}{k} \\
  n &= \frac{\symup{c}}{v_{\text{ph}}}
\end{align*}
zwischen Phasengeschwindigkeit und Brechungsindex wird Gleichung \eqref{eqn:theorietheta}
\begin{equation*}
  θ = \frac{L ω}{2 \symup{c}} \l(n_R - n_L\r)\,.
\end{equation*}
Die Abhängigkeit des Winkels von den Brechungsindizes ist klar zu sehen.

\subsection{Betrachtung eines Elektrons}
Startet man bei der Bewegungsgleichung
\begin{equation*}
  m \frac{\partial^2 \vec{r}}{\partial t^2} + K \vec{r} =
  - \me_0 \vec{E}\l(\vec{r}\r) - \me_0 \frac{\ud\vec{r}}{\ud t} \times \vec{B}
\end{equation*}
kann mittels des Polarisationsvektors $\vec{P}$ und der dielektrischen
Suszeptibilität $χ$ die Gleichung \eqref{eqn:theorietheta}
erweitert werden zu
\begin{equation*}
  θ  = \frac{\me_0^3}{2 ε_0 \symup{c}} \frac{ω^2}{\l(-mω^2+K\r)^2 - \l(\me_0ωB\r)^2}
    \frac{N B L}{n}\,.
\end{equation*}
Hierbei bezeichnet $ω$ die Frequenz des Feldes, $K$ eine Bindungskonstante im
Festkörper und $N$ die Zahl der Elektronen pro Volumeneinheit \cite{anleitung}.

Bei der weiteren Vereinfachung des Terms werden die Frequenzen betrachtet.
Wenn die Frequenz mit der gemessen wird, nicht zu nah an der Zyklotron-Frequenz
\begin{align*}
  ω_c &= \frac{B \me_0}{m}
  \intertext{liegt, gilt}
  \l(ω_0^2 - ω^2\r)^2 ≫ ω^2ω_c^2\,.
\end{align*}
Bei $ω ≪ ω_0: \l(ω_0^2 - ω^2\r)^2 = ω_0^4$.
Mit der Wellenlänge $λ$ kommt man zu der Gleichung
\begin{equation*}
  θ\l(λ\r) = \frac{2 \mpi^2 \me_0^3 \symup{c}}{ε_0 m^2} \frac{1}{λ^2 ω_0^4}
    \frac{N B L}{n}\,.
\end{equation*}
Mit dem Grenzwert $ω_0 \to 0$ kann man freie Ladungsträger betrachten.
Für Kristallelektronen gilt dies, unter bestimmten Voraussetzungen \cite{anleitung},
bei der Ersetzung von $m$ mit der effektiven Masse $m^\ast$.
Es folgt
\begin{equation}
  θ_{\text{frei}} = \frac{\me_0^3}{8 \mpi^2 ε_0 \symup{c}^3} \frac{λ^2}{\l(m^\ast\r)^2}
    \frac{N B L}{n}\,. \label{eqn:thetafrei}
\end{equation}
