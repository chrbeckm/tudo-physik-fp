\section{Diskussion}
\label{sec:Diskussion}
Die Messung des Magnetfeldes der Spule liefert ein Ergebnis,
welches für eine Spule mit Ferromagnetikum nicht ungewöhnlich ist.

In den Messungen der jeweiligen Winkel war eine ganz genaue Einstellung
der Winkel nicht möglich.
Auf dem Oszilloskop-Schirm war in einem breiten Winkelbereich,
meist $\SI{2}{\degree}$, keine Veränderung zu erkennen.
Die Minima wiederholten sich jedoch in vielfachen von $\SI{90}{\degree}$.
Dass die Minima ausgeschmiert sind, kann daran liegen, dass die Öffnung
in der Spule nicht komplett in der optischen Achse liegt, und somit nicht die
komplette Intensität genutzt werden kann.

Die errechneten effektiven Massen freier Elektronen weichen von den Literaturwerten ab.
Die Unterschiede sind auf den oben genannten Winkelbereich zurückzuführen.