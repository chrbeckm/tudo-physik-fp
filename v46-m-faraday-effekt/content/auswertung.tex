\section{Auswertung}
\label{sec:Auswertung}
\subsection{Magnetfeld der Spule}
Die Messwerte dieser Messung befinden sich im Anhang.
Das Maximum befindet sich bei
\begin{align*}
  B_{\text{max}} &= \SI{422}{\milli\tesla} \\
  x_{\text{max}} &= \SI{109}{\milli\meter}\,.
\end{align*}
\begin{figure}
  \centering
  \includegraphics[width=0.8\textwidth]{build/p-messung-a.pdf}
  \caption{Graphische Darstellung des Magnetfeldes innerhalb der Spule am Ort der Proben.}
  \label{fig:messung-a}
\end{figure}

\subsection{Faraday-Effekt}
Aus den genommen Winkeln werden die Differenzen zwischen den beiden
Polrichtungen des Magnetfeldes bestimmt
\begin{equation*}
  θ_{\text{kr}} = \frac{\abs{θ_1 - θ_2}}{2}
\end{equation*}
die Winkel $θ_{\text{kr}}$ werden dann durch die jeweilige Dicke der Probe geteilt.
Das Ergebnis ist in Abbildung \ref{fig:messung-b} zu sehen.
\begin{figure}
  \centering
  \includegraphics[width=0.8\textwidth]{build/p-messung-b.pdf}
  \caption{Graphische Darstellung der Faraday Winkel.}
  \label{fig:messung-b}
\end{figure}
\FloatBarrier
In Abbildung \ref{fig:messung-b-diff} sind die Differenzen zwischen den Faraday-Winkeln
der Proben 1, 2 und der reinen Probe 0 dargestellt.
\begin{figure}
  \centering
  \includegraphics[width=0.8\textwidth]{build/p-messung-b-diff.pdf}
  \caption{Graphische Darstellung der Faraday Winkel minus der Winkel der reinen Probe.}
  \label{fig:messung-b-diff}
\end{figure}
Die Ausgleichskurven der Form
\begin{equation*}
  θ\l(λ\r) = a \cdot λ^2
\end{equation*}
haben mit \texttt{scipy} die Faktoren
\begin{align*}
  \input{build/p-messung-b-a1.tex} \\
  \input{build/p-messung-b-a2.tex}\,.
\end{align*}
Nach Gleichung \eqref{eqn:thetafrei} ist
\begin{align*}
  a_i &= \frac{\me_0^3}{8\mpi^2 ε_0 \symup{c}^3} \frac{N_i B}{n} \frac{1}{\l(m^\ast\r)^2} \\
  m^\ast &= \sqrt{\frac{\me_0^3 B}{8 \mpi^2 ε_0 \symup{c}^3 n}\frac{N_i}{a_i}}
\end{align*}
mit den Konstanten \cite{constants}
\begin{align*}
  \me_0 &= \SI{1.6021766208e-19}{\coulomb} \\
  B &= \SI{423}{\milli\tesla} \\
  ε_0 &= \SI{8.854187817620389e-12}{\ampere\second\per\volt\per\meter} \\
  \symup{c} &= \SI{299792458}{\meter\per\second}
  \shortintertext{\cite{gaasn}}
  n &= 3.57\,.
\end{align*}

Die Massen folgen zu
\begin{align*}
  \input{build/p-messung-b-m1.tex} \\
  \input{build/p-messung-b-m2.tex}\,.
  \intertext{Der Literaturwert ist}
  m^\ast &= \num{0.063} m_{\me} = \SI{0,10414148e-31}{\kilo\gram}.
\end{align*}
Werden nur die positiven Werte in den Fits verwendet, folgen die effektiven Massen zu
\begin{align*}
  m_1^\ast &= \SI{0.59(8)e-31}{\kilo\gram} \\
  m_2^\ast &= \SI{0.89(9)e-31}{\kilo\gram}\,.
\end{align*}

\begin{table}
  \centering
  \caption{Umgerechnete Messwerte der \textsc{Faraday}-Winkel.}
  \label{tab:messwerte}
  \sisetup{table-format=1.2}
  \begin{tabular}{S[table-format=1.3] S S S}
    \toprule
    {Filter\:/\:$\si{\micro\meter}$} &
    {Rein\:/\:$\si{\radian\per\meter}$} &
    {$\uD$Probe~1\:/\:$\si{\radian\per\meter}$} &
    {$\uD$Probe~2\:/\:$\si{\radian\per\meter}$} \\
    \midrule
    1.060 & 56.36 & -40.31 & -44.01 \\
    1.290 & 35.86 & -31.59 &   5.10 \\
    1.450 & 18.50 &  -3.53 &   7.87 \\
    1.720 & 18.50 & -14.22 & -10.64 \\
    1.960 & 10.25 &  27.18 &  20.05 \\
    2.156 &  9.96 &  23.19 &  36.05 \\
    2.340 & 20.21 & -15.93 &  15.70 \\
    2.510 &  3.53 &  90.58 &  97.47 \\
    2.650 & 16.79 &  45.23 &  72.99 \\
    \bottomrule
  \end{tabular}
\end{table}

