\subsection{Wasserstoffmolekül}
\label{sec:wasserstoffmolekül}

Die Werte der Messung der Resonanzfrequenzen in Abhängigkeit der zwischengelegten Blende stehen
in Tabelle \ref{tab:h2blende} und sind in Abbildung \ref{fig:h2blende} aufgetragen.

\begin{table}
  \centering
  \caption{Resonanzfrequenzen der verschiedenen Blenden.}
  \label{tab:h2blende}
  \begin{tabular}{S[table-format=2.0] S[table-format=1.3]}
    \toprule
    {$d_\text{Blende}\:/\:\si{\milli\meter}$} & {$f_\text{res}\:/\:\si{\kilo\hertz}$} \\
    \midrule
    10 & 2.135 \\
    13 & 2.135 \\
    16 & 2.134 \\
    25 & 2.13 \\
    \bottomrule
  \end{tabular}
\end{table}

\begin{figure}
  \centering
  \includegraphics[width=0.8\textwidth]{build/h2blende.pdf}
  \caption{Resonanzfrequenzen der verschiedenen Blenden aufgetragen gegen die Blendenstärke.}
  \label{fig:h2blende}
\end{figure}
\FloatBarrier
Abbildung \ref{fig:h2winkel} zeigt die Winkelabhängigkeit der Druckamplitude für
$\SI{2.135}{\kilo\hertz}$.
Die Quantenzahlen sind
\begin{align}
  l &= 0 \\
  m &= 0\:.
\end{align}
Dies entspricht der Kugelflächenfunktion $Y_{00}$, Gleichung
\eqref{eqn:y00}.
\begin{figure}
  \centering
  \includegraphics[width=0.8\textwidth]{build/h2winkel.pdf}
  \caption{Winkelverteilung des Wasserstoffmoleküls mit der $\SI{10}{\milli\meter}$-Blende.}
  \label{fig:h2winkel}
\end{figure}
\FloatBarrier
