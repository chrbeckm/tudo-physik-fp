\section{Diskussion}
\label{sec:Diskussion}
Die in der ersten Messaufgabe bestimmte Schallgeschwindigkeit
\begin{align}
  c &= \SI{350(3)}{\meter\per\second}
  \intertext{ist in der Nähe des Literaturwertes \cite{SpeedOfSound}}
  c_\text{lit} &= \SI{343}{\meter\per\second}\,,
  \intertext{jedoch mehr als 3 Standardabweichungen daneben.
  Eine weitere Abweichung ist der Parameter}
  a &= \num{0.984(3)}\,.
  \intertext{Eine Dimensionsanalyse mit Gleichung \eqref{eqn:resonanzfrequenz} zeigt schnell, dass}
  a &= 1
\end{align}
gelten muss.
Mögliche Fehlerquellen sind hier die Genauigkeit der Frequenz
und die nicht betrachtete Abhängigkeit der Schallgeschwindigkeit von Temperatur und Druck.

Die Bestimmung des Frequenzspektrums der anwachsenden Zylinderkette zeigt die in der Theorie
vorhergesagten Effekte.
Mit steigender Zylinderanzahl erhöht sich die Anzahl der Resonanzen in einem gleichbleibenden
Frequenzbereich, wie es Gleichung \eqref{eqn:resonanzfrequenz} vorhersagt.
Zu sehen ist dies in Abbildung \ref{fig:50mm-computer-04-12}
Oszilloskopbilder und Computermessungen stimmen überein,
wie es Abbildung \ref{fig:50mm-oszi-computer-01} verdeutlicht.
\\~\\
Bei der Resonanzfrequenzmessung des Wasserstoffatoms stimmen die Ergebnisse des Oszilloskops und
des Computers nicht überein, wie es in Tabelle \ref{tab:h-resonanz} steht.
Aus der Computermessungen kann keine Phase gewonnen werden, auf dem Oszilloskopschirm
kann die Phase aus den Lissajousfiguren bestimmt werden, diese zeigen auch an,
dass eine Resonanz vorliegt.
Die Einstellmöglichkeiten der Frequenz am Sinusgenerator ließen allerdings nicht zu,
die genaue Figur zu sehen, was auch zu Unsicherheiten führt, aber auch Figuren überspringen kann.

Die Winkelaufgelste Messung der Druckamplitude soll in Übereinstimmung mit den
Kugelflächenfunktionen sein, die den Winkelanteil der jeweiligen Differentialgleichung lösen.
Für die erste Resonanzfrequenz passen Theorie und Messwerte nicht ganz übereinander,
die Theorie gibt einen konstanten Wert vor, im Experiment messen wir einen Anstieg proportional
zum Winkel.
Für die zweite Resonanz passen Theorie und Messwerte sehr gut überein.
Die Messwerte der dritten Resonanz passen weder mit den Theoriewerten der $l=2$ oder $l=3$
Kugelflächenfunktion überein. Die Kugelflächenfunktion mit $l=4$ passt bis auf die Amplitudenstärke gut mit den Messwerten überein,
die Position der Minima und Maxima passen bis auf wenige Grad übereinander.
Hier wurden folglich zwei Resonanzen nicht im Frequenzspektrum, Abbildung \ref{fig:h-resonanz}, gesehen.

Die Messung der Aufspaltung der Peaks bei verschieden dicken Zwischenringen liefert kein
eindeutiges Ergebnis wo die Peaks für die einzelnen Ringstärken liegen.
Dies ist gut in Abbildung \ref{fig:zwischenringe-zoom} zu erkennen.
Die Messwerte liefern keine Möglichkeit mehrere Peaks zu erkennen.

Die Winkelmessung mit dem $\SI{9}{\milli\meter}$ Zwischenring ergibt eine Form im Polarplot
\ref{fig:9mm} die an Abbildung \ref{fig:h-381} erinnert.
Aussagen über die zu den Resonanzen gehörigen Quantenzahlen $l$ und $m$ können aufgrund von
fehlenden Resonanzen nicht gemacht werden.

Die aufgetretenen Fehler sind schwer zu lokalisieren, da das Wasserstoffatom ein
zusammenhängendes System ist.
Einige Messungen liefern hier auch mit der Theorie übereinstimmende Ergebnisse,
während andere mit veränderten Parametern, oder anderen Teilen dies nicht tun.
\\~\\
Die Messung der Abhängigkeit der Resonanzfrequenz von der Blendenstärke des
Wasserstoffmoleküls liefert kein klares Ergebnis.
Die Anzahl und Unterschiede zwischen den Blenden ist zu klein um in
Kombination mit der $\SI{1}{\hertz}$-Schrittweite des Frequenzgenerators
verwertbare Ergebnisse zu liefern.
Die Messung mit $d_\text{Blende} = \SI{25}{\milli\meter}$, welche deutlich
von den anderen abweicht wurde ohne Blende durchgeführt.
Die Winkelaufgelöste Messung widerspricht der Theorie klar, da hier eine
nahezu konstante Amlitude zu messen ist.
\\~\\
Aus den Frequenzspektren des eindimensionalen Festkörpers für unterschiedliche Anzahlen von Zylindern können keine Unterschiede festgestellt werden.
Dies entspricht exakt der theoretischen Vorhersage in Gleichung \eqref{eqn:omega1a}, da $ω$ nicht von der Länge des Systems abhängt.
Auch bei den Spektren mit der $\SI{10}{\milli\meter}$ Blende gibt es keine Unterschiede zwuschen den Kurven.
Bemerkenswert ist hier, dass der Blendenwechsel tatsächlich einen Unterschied für $f_\text{res}$ macht,
die Blendestärke ist vermutlich mit der Kopplungskonstanten $c$ verknüpft.
Ein genauerer Zusammenhang ist aufgrund der Schrittweite nicht möglich.
\\
Das Austausch einer der 12 Zylinder gegen einen mit einer Länge von $\SI{75}{\milli\meter}$
führt zu einer, wenn auch geringfügigen, Verschiebung des Maximums.
Bei der Ersetzung dieses Zylinders durch zwei $\SI{12.5}{\milli\meter}$ Zylinder,
verschiebt sich das Maximum nocheinmal.
Wie groß der Unterschied zwischen der Messung mit Blende zwischen den kurzen Zylindern und ohne ist,
kann aufgrund der statistischen Fluktuationen in Verbindung mit der Schrittweite nicht gesagt werden.
%Sowohl in der grünen als auch in der blauen Kurve, bei welcher ein Zylinder mehr in der Kette ist, gab es eine Aufspaltung der Peaks. Es kann sich aber auch um statistische Fluktuation handeln, da unsere Schrittweite der Messungen $\SI{10}{\hertz}$ betrug.
\\
Bei der Frequenzspektrumsmessung der alternierenden Blenden mit $\SI{13}{\milli\meter}$ und $\SI{16}{\milli\meter}$ Durchmesser wurde zwei mal ein sehr ähnliches Spektrum aufgenommen.
Wie auch schon aus den einheitlichen Messreihen der Blenden hervorgeht ist der Unterschied zwischen den Frequenzen sehr gering, sodass die umkehr der Reihenfolge kein bedeutender Faktor ist.
%Uns ist nicht klar warum, möglicherweise war unsere Schaltung nicht korrekt aufgebaut.
\newpage
\textbf{Korrektur:}
Die Frequenzspektren der Zylinderkette mit unterschiedlichen Längen und Blenden
zeigen keine klaren Resonanzfrequenzen.
Dies kann daran liegen, dass die Schallwellen innerhalb eines Zylinders oder mehreren Zylindern
reflektiert werden. Die Positionen der Resonanzfrequenen sind abhängig von den Blenden,
die Theorie gibt es mit Gleichung \eqref{eqn:omega1a} vor. Wenn $c$ mit der Blendenstärke
identifiziert wird.
\\~\\
Die Messdaten der Zylindertauschmessung zeigen, dass das Frequenzspektrum
abhängig von der Position der Störstelle ist.
Die Änderungen sind zwar gering, treten aber bei beiden Zylindern gleichermaßen auf.
Die Theorie gibt keine Abhängigkeit von der Position der Störstelle an.
Es wird jedoch immer mit unendlichen Ketten gerechnet, inwieweit die Enden selber
und dann die Position der Störstelle einen Einfluss haben ist daher unbekannt.
