\section{Durchführung}
\label{sec:Durchführung}
In der ersten vorbereitenden Messaufgabe wird eine länger werdende Zylinderkette vermessen,
indem zuerst die Frequenzen zweier aufeinanderfolgenden Resonanzen mit dem Oszilloskop gemessen werden.
Die Eingangsfrequenz wird durch einen Sinusgenerator und einen Lautsprecher in die
Zylinderkette gebracht und über ein Mikrofon auf der gegenüberliegenden Seite detektiert.
Die Messung beginnt mit einem Zylinder der Länge $L = \SI{50}{\milli\meter}$,
dann wird immer ein weiterer Zylinder angereiht, bis es zwölf sind.

Für die anwachsende Zylinderkette wird dann ebenfalls das Frequenzspektrum,
jeweils per Oszilloskop und Computer, im Bereich von $\SI{1}{\kilo\hertz}$ bis
$\SI{10}{\kilo\hertz}$, aufgenommen.
\\~\\
Der zweite Versuchsteil beschäftigt sich mit dem Wasserstoffatom-Modell.
Dieses besteht aus zwei Kugelhälften, in der unteren Kugelhälfte befindet sich der
Lautsprecher, in der oberen das Mikrofon.
Setzt man das Modell zusammen und betrachtet Kugelkoordinaten,
befindet sich der Lautsprecher bei $θ_L = \sfrac{3\mpi}{4}$
und das Mikrofon bei $θ_M = \sfrac{\mpi}{4}$.
Der Winkel zwischen den beiden Elementen kann außen als Winkel $α$ abgelesen werden.

In der ersten Messung ist $α = \SI{180}{\degree}$ gesetzt
und es werden die Resonanzen des Sinus-Eingangs-Signals zwischen
$\SI{100}{\hertz}$ und $\SI{10}{\kilo\hertz}$ gesucht.
Auch hier mit Oszilloskop und Computer.

Als nächstes wird das entsprechende kleinschrittige Frequenzspektrum bei
$α = \SI{180}{\degree}$ aufgenommen.
Von drei dieser Resonanzfrequenzen wird ein Winkelspektrum in
$\SI{10}{\degree}$-Schritten aufgenommen.

Die zwei Zwischenringe, die zwischen die Kugelhälften passen,
werden in allen 4 Kombinationen eingesetzt und es wird je ein Frequenzspektrum
in der Umgebung von $\SI{2130}{\hertz}$ genommen.

Für die Kombination aus Ringen die eine Dicke von $\SI{9}{\milli\meter}$ ergeben
wird eine Winkelauflösungsmessung bei einer Frequenz von $\SI{2130}{\hertz}$
durchgeführt.
\\~\\
Der dritte Versuchsteil befasst sich mit dem Wasserstoffmolekül.
Für dieses werden zwischen die Kugelhälften des Wasserstoffmoleküls
zwei weitere Kugelh\"lften gepackt, diese haben ein Loch über das die
einzelnen Kugelresonatoren akustisch gekoppelt werden.

Bei der ersten Messung wird je eine der drei Blenden zwischen die Kugeln gepackt
und die Resonanzfrequenz im Bereich um $\SI{2130}{\hertz}$ gesucht.

Für die $\SI{10}{\milli\meter}$-Blende wird dann eine Winkelmessung bei einer
Frequenz von $\SI{2135}{\hertz}$ vollzogen.
\\~\\
Der vierte Versuchsteil befasst sich mit dem eindimensionalen Festkörper.
Hier wird wieder wie im ersten Teil eine Zylinderkette stückweise aufgebaut,
allerdings in der ersten Runde mit $\SI{10}{\milli\meter}$-Blenden,
beim zweiten Mal mit $\SI{13}{\milli\meter}$-Blenden.
Es wird jeweils ein Frequenzspektrum von $\SI{1}{\kilo\hertz}$ bis
$\SI{4}{\kilo\hertz}$ aufgenommen.

In der fertigen Kette wird der Zylinder an Position 6 mit einem
$\SI{75}{\milli\meter}$ langen Zylinder ausgetauscht.
Nach der Messung des Frequenzspektrums wird dieser durch zwei
$\SI{12.5}{\milli\meter}$ lange Zylinder ausgewechselt.

Als letzte Messung wird eine Kette aus zwölf $\SI{50}{\milli\meter}$ Zylindern
und abwechselnden Blenden mit $\SI{13}{\milli\meter}$ und $\SI{16}{\milli\meter}$
aufgebaut. Die äußeren Blenden werden nach der Frequenzmessung gewechselt,
sodass die Reihenfolge von $13-16-13$ auf $16-13-16$ wechselt.
