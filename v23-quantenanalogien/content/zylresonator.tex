\subsection{1-dimensionaler Festkörper}
\label{sec:1dimFK}
\subsubsection{1-dimensionale Kette mit $\SI{13}{\milli\meter}$ Blende}
Aus den Frequenzspektren für die verschieden langen Zylinderketten mit
$\SI{13}{\milli\meter}$ Blenden, werden die Positionen der Maxima in Abhängigkeit der Zylinderanzahl bestimmt. Die Werte sind in Tabelle \ref{tab:13mm} aufgelistet.
In der unten stehen Grafik \ref{fig:13mmblende} werden beispielhaft die Frequenzspektren
für 1, 4 und 12 Zylinder aufgetragen.
Die Messwerte sind leicht zueinander in der Amplitude (y-Koordinate) verschoben,
da sonst nicht erischtlich ist wo die einzelnen Werte liegen.

\begin{table}
	\centering
	\caption{Abhängigkeit der Resonanzen von der Zylinderanzahl bei $\SI{13}{\milli\meter}$ Blenden.}
	\label{tab:13mm}
	\sisetup{table-format=4.0}
	\begin{tabular}{c SSS SSS SSS SSS}
		\toprule
		{\#Zylinder} & 1 & 2 & 3 & 4 & 5 & 6 & 7 & 8 & 9 & 10 & 11 & 12 \\
		\midrule
		{$f_\text{res}\:/\:\si{\hertz}$} & 2140 & 2150 & 2140 & 2140 & 2140 & 2140 & 2140 & 2150 & 2140 & 2150 & 2150 & 2130 \\
	\bottomrule
	\end{tabular}
\end{table}
%Da die Unterschiede zwischen den Maxima zum einen sehr klein sind und zum
%zweiten sehr statistisch aussehen, kann über die Abhängigkeit der
%Resonanzpositionen zur Resonatorlänge (Anzahl an Zylindern) keine Aussage gemacht werden. \\

\begin{figure}
	\centering
		\includegraphics[width=0.8\textwidth]{build/zylinder1_4_12-13mm-Blende.pdf}
	\caption{Frequenzspektrum für die Zylinderkette mit 13mm Blende.}
	\label{fig:13mmblende}
\end{figure}
\FloatBarrier

\subsubsection{1-dimensionale Kette mit $\SI{10}{\milli\meter}$ Blende}
Analog zur Messung mit einer $\SI{13}{\milli\meter}$ Blende, ist auch bei
der $\SI{10}{\milli\meter}$ Blende keine Korrelation zwischen den
Positionen der Resonanzen und der Zylinderanzahl erkennbar.
In Tabelle \ref{tab:10mm} sind die Messwerte für die Positionen der Maxima
aufgetragen.
\begin{table}
	\centering
	\caption{Abhängigkeit der Resonanzen von der Zylinderanzahl bei $\SI{10}{\milli\meter}$ Blenden.}
	\label{tab:10mm}
	\sisetup{table-format=4.0}
	\begin{tabular}{c SSS SSS SSS SSS}
		\toprule
		{\#Zylinder} & 1 & 2 & 3 & 4 & 5 & 6 & 7 & 8 & 9 & 10 & 11 & 12 \\
		\midrule
		{$f_\text{res}\:/\:\si{\hertz}$} & 2150 & 2140 & 2140 & 2140 & 2150 & 2150 & 2140 & 2150 & 2140 & 2140 & 2150 & 2150 \\
	\bottomrule
	\end{tabular}
\end{table}

\begin{figure}
	\centering
	\includegraphics[width=0.8\textwidth]{build/zylinder1_4_12-10mm-Blende.pdf}
	\caption{Frequenzspektrum für die Zylinderkette mit 10mm Blende.}
\end{figure}

\newpage

% Das ist die erste Messaufgabe.
%
%\subsubsection{1-dimensionale Kette ohne Blende}
%
%\begin{figure}
%	\centering
%	\includegraphics[width=0.8\textwidth]{build/zylinder1_4_12-ohne-Blende.pdf}
%	\caption{Frequenzspektrum für die Zylinderkette ohne Blende.}
%	\label{fig:ketteOhne}
%\end{figure}
%
%In Abbildung \ref{fig:ketteOhne} sind exemplarisch die Frequenzspektren für 1, 4 und 12
%Zylinder, ohne Blenden, aufgetragen. Es ist klar zu erkennen, dass bei einem Zylinder
%3 Resonanzen auftreten. Demnach sind in dem Spektrum für 4 Zylinder 12 Resonanzen zu erkennen und im Spektrum mit 12 Zylindern 36 Resonanzen.
%Dies ist konsistent mit der Theorie nach \eqref{eqn:resonanzfrequenz}.
%
%\newpage

\subsubsection{Zylinderaustausch}
In Abbildung \ref{fig:austausch} sind die Spektren mit den verschiedenen
Zylindervertauschungen aufgetragen.
In Abbildung \ref{fig:peaks} ist die Umgebungen um die Aufspaltung vergrössert.
Die Positionen der globalen Maxima der Kurven sind in Tabelle \ref{tab:einschubMaxima} aufgetragen.
Die Schrittweite der Messungen beträgt $\SI{10}{\hertz}$.
\begin{figure}
	\centering
	\includegraphics[width=0.8\textwidth]{build/kette_2.pdf}
	\caption{Austausch verschiedener Zylinder innerhalb der Kette.}
	\label{fig:austausch}
\end{figure}

\begin{figure}
	\centering
	\includegraphics[width=0.8\textwidth]{build/peaks_kette.pdf}
	\caption{Auspaltung der Peaks bei den verschiedenen Zylindern.}
	\label{fig:peaks}
\end{figure}

\begin{table}
	\centering
	\caption{Austausch eines Zylinders gegen unterschiedlich lange andere Zylinder.}
	\label{tab:einschubMaxima}
	\begin{tabular}{c S[table-format=4.0]}
		\toprule
		{Zylindereinschub} & {relative Position der Maxima} \\
		\midrule
		2 \times \SI{12.5}{\milli\meter} \text{ohne Blende} & 9240 \\
		2 \times \SI{12.5}{\milli\meter} \text{mit Blende}  & 9220 \\
		\SI{75}{\milli\meter} & 9190 \\
		\bottomrule
	\end{tabular}
\end{table}

\newpage
\subsubsection{Alternierende Blenden}
In Abbildung \ref{fig:alt} sind die beiden Varianten an Blendenkonstellationen zusehen.
Für die schwarze Kurve sind die äußersten Blenden die mit $\SI{13}{\milli\meter}$,
für die rote genau andersrum.
% Wie man sehen kann, liegen beide Kurven bis auf einige statistische Schwankungen, exakt übereinander.
\begin{figure}
	\centering
	\includegraphics[width=0.7\textwidth]{build/abwechselnde-Blende.pdf}
	\caption{Frequenzspektrum für alternierende Blenden innerhalb der Kette.}
	\label{fig:alt}
\end{figure}
