\section{Diskussion}
\label{sec:Diskussion}

Am aufgenommenen Spektrum der Quelle lassen sich gut die verschiedenen Phänomene ablesen. Jegliche Unsicherheiten sind statistischer Natur und durch die Anzahl an Bins gegeben.
Der Peak der aufgrund der Cs-Quelle zu erwarten ist, ist klar ausgebildet.
In einem großen Energieplateau davor sieht man den Teil der Strahlung, der durch die \glname{Compton}{-Streuung} zustande kommt.
\\~\\
Die Vermessung des Aluminiumgehäuses, bringt eine Abschwächungskonstante, die stark vom Literaturwert abweicht.
Dies kann daran liegen, dass die Luft nicht mit einbezogen wurde, obwohl diese $γ$-Strahlung nahezu nicht abschwächen kann.
Ein weiterer Faktor ist die Hülle mit $\SI{2}{\milli\meter}$-Stärke,
diese hat kaum einen Einfluss.

Der Referenzwürfel~2 kann Messing zugeordnet werden,
der relative Fehler beträgt $f_2\approx\SI{2}{\percent}$.

Die Auswertung des Referenzwürfels~3 bringt keine eindeutige Abschwächungskonstante.
Er wird mit einem relativen Fehler von
$f_3\approx\SI{87}{\percent}$ als aus Eisenwürfeln bestehend zugeordnet.
Kein anderes vorgegebenes Material passt besser.

Die Teilwürfel des Würfel~5 passen mit den Referenzmessungen einigermaßen zusammen.
Die Abschwächungskonstanten der Messingwürfel sind kleiner als in der Vergleichsmessung,
die der Eisenwürfel größer, die relativen Fehler zum Literaturwert sind dennoch sehr groß.
\\~\\
An der Messung kann die Positionierung und Ausrichtung der Würfel im Strahlengang verbessert werden.
Zudem kann die Anzahl an gemessenen Intensitäten erhöht werden.
