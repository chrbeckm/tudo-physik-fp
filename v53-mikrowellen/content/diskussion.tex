\section{Diskussion}
\label{sec:Diskussion}
Bei der Vermessung der Moden, kann man in Abbildung \ref{fig:frequenzbild}
können die Moden bei $\SI{115}{\volt}$, $\SI{160}{\volt}$ und eine
Überlagerung von $\SI{190}{\volt}$ und $\SI{235}{\volt}$ beobachtet werden.
Auch die Bestimmung der Länge des Klystrons liefert ein konsistentes Ergebnis.
\\~\\
Die Frequenzmessung mit der Reflektorspannung als Parameter zeigt,
dass bei höheren Spannungen die Ausgangsfrequenz geringer wird.
Die Amplitude der Mikrowellen steigt mit der Reflektorspannung an.
Dieses liegt daran, dass die Elektronen mit höherer Energie in die Kavität gelangen.
\\~\\
Die Bestimmung der Frequenz in der zweiten Messreihe zeigt,
dass der Unterschied in der Hohlleiterlänge einen Einfluss auf die Wellen hat.
Die Phasengeschwindigkeit zeigt deutlich, dass es sich um Elektromagnetsche Wellen handelt,
die sich mit Lichtgeschwindigkeit ausbreiten.
Für die Phasengeschwindigkeit ist es durchaus erlaubt größer\, als die
Lichtgeschwindigkeit im Vakuum zu sein.
Sowohl die berechneten Wellenlängen als auch die Frequenzen, welche mit ca.
$\SI{9}{\giga\hertz}$ Mikrowellen beschreiben, sind konsistent.
\\~\\
Die Bestimung der Dämpfung des Mikrowellenfeldes zeigt einen systematischen Fehler zwischen
Theoriewerten und den Messwerten.
Die Fitparamter zeigen zudem auch große Fehler, dies ist in der Abbildung \ref{fig:daempfbild}
auch zu sehen.
Die e-Funktion passt an den Rändern nicht zu den Messwerten.
\\~\\
Bei der direkten Methode zeigt sich, dass möglicherweise das SWR-Meter statistische
Ungenauigkeiten aufweist, da bei einer Stifttiefe von $\SI{0}{\milli\metre}$ ein SWR von 1
auftreten sollte, genauso sollte bei einer Stifttiefe von $\SI{7}{\milli\metre}$ das SWR
wesentlich größer \,sein als 1.

Die Korrekturwerte, die wir für die Auswertung herangezogen haben, zeigen
deutlich realistischere Ergebnisse; Bei $\SI{0}{\milli\metre}$ Stifttiefe soll
das SWR gerade 1 sein, und bei $\SI{9}{\milli\metre}$ hat das SWR-Meter einen
Werte von 5 angezeigt, obwohl es eigentlich unendlich sein sollte.
Das liegt einfach daran, dass bei $\SI{9}{\milli\metre}$ die Schwankung des
SWR so groß\, ist, dass einfach kein genauer Wert abgelesen werden kann.

Wir haben für die letzten beiden Methoden den festen Kurzschluss verwendet, welcher analog zu einem Gleitschraubentransformator mit einer Stifttiefe von $\SI{9}{\milli\metre}$ ist.
Ein SWR von ca. 8, wie es bei der 3dB-Methode der Fall war, ist
plausibel, da bei einem festen Kurzschluss das SWR sehr groß\, ist.

Unsere berechneten Werte bei der Abschwächer Methode zeugen von einer großen\,
Schwierigkeit genaue Dämpfungen einzustellen. Wohin gegen die Korrekturwerte
ein SWR von $\num{7.08}$ ergeben. Dies ist plausible aus den selben Gründen
wie bei der 3dB-Methode.
