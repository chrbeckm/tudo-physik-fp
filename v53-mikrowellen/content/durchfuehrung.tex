\section{Aufbau}
\label{sec:Aufbau}
Die Mikrowellen werden in einem Reflexklystron erzeugt.
Dieses wird von einem Netzgerät betrieben, dass entweder eine $\SI{50}{\hertz}$-Sinus oder
$\SI{1}{\kilo\hertz}$-Rechteck Spannung liefert.
Die Mikrowellen treten aus dem Reflexklystron in den Hohlleiter über.
Der anschließende Einweggleichrichter stellt sicher, dass die Erzeugung im Reflexklystron nicht durch die reflektierten
Wellen gestört wird.
Der Frequenzmesser ist analoger Bauweise, durch einen Resonator, der in der Größe verändert werden kann,
wird ein Teil der Leistung aus der Welle genommen, dieses kann durch eine Delle auf dem Oszilloskop erkannt werden.
Es folgt der Foliendämpfer, wenn er nicht verwendet wird, ist die Folie soweit zurückgezogen,
dass sie nicht im Profil des Hohlleiters ist.
Die folgenden Teile variieren je nach Messaufgabe.

\section{Durchführung}
\label{sec:Durchführung}
In der ersten Messung wird hinter den oben beschriebenen Aufbau eine Diode geschraubt.
Der Foliendämpfer wird auf $\SI{30}{\deci\bel}$ gestellt.
Das Ausgangssignal der Diode wird gegen die an das Reflexklystron angelegte Spannung auf dem Oszilloskop dargestellt.
Das Netzgerät wird mit der Option 0-$\SI{30}{\volt}$, $\SI{50}{\hertz}$ betrieben.
Es wird die Frequenz und Ausgangsleistung in Abhängigkeit der Reflektorspannung gemessen.
\\~\\
Für die zweite Messung wird statt der Diode ein verschiebbarer Stehwellendetektor und ein einstellbarer Kurzschluss verwendet.
Bei einer Reflektorspannung von $\SI{200}{\volt}$ und einer Dämpfung von $\SI{20}{\deci\bel}$, mit der $\SI{1}{\kilo\hertz}$
Modulation, werden die Abstände zweier Maxima bzw. Minima bei festem Kurzschluss bestimmt.
\\~\\
Die dritte Messreihe befasst sich mit der Dämpfung. Hierfür wird der verstellbare Kurzschluss durch einen Abschluss ersetzt.
Gemessen wird die Mikrowellenleistung, mit einem SWR-Meter,
in Abhängigkeit der Stellung der Mikrometerschraube am Foliendämpfer.
\\~\\
In der vierten und letzten Messung wird zwischen Detektor und Abschluss ein Gleitschraubentransformator eingesetzt.
Dieser zeichnet sich dadurch aus,
dass ein Stift mittig im Hohlleiter abgesenkt und die Position in Ausbreitungsrichtung der Wellen verändert werden kann.
Es werden drei verschiedene Methoden verwendet.

In der ersten Methode wird die Stifttiefe 0, 3, 5 oder $\SI{7}{\milli\meter}$ betragen.
Gemessen wird das Stehwellenverhältnis direkt mit dem SWR-Meter.

Für die zweite Methode wird ein Minimum gesucht.
Rechts und links wird der Abstand gesucht, bei dem die Amplitude das Doppelte des Minimums ist.

Die letzte Methode ist die Abschwächer-Methode.
Es werden zwei Dämpfungen gesucht, bei denen die Leistung im Minimum bei kleinerer Dämpfung und im Maximum bei geringerer
Dämpfung gleich sind.
