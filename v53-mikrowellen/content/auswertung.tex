\newpage
\section{Auswertung}
\label{sec:Auswertung}
\subsection{Untersuchung der Moden}
In Abbildung \ref{fig:frequenzbild} ist die Frequenz der Mikrowellen gegen die Spannung am Reflektor aufgetragen.
Die Abhängigkeit der Mikrowellenamplitude von der Reflektorspannung ist in Abbildung \ref{fig:Uausbild} gezeigt.
Die Mikrowellenamplitude wird als Spannung dargestellt, da es mit einer Diode gemessen wird.
\begin{figure}
  \centering
  \includegraphics[width=0.6\textwidth]{build/messung1_f.pdf}
  \caption{Plot der Abhängigkeit der Frequenz von der Reflektorspannung.}
  \label{fig:frequenzbild}
\end{figure}
\begin{figure}
  \centering
  \includegraphics[width=0.6\textwidth]{build/messung1_Uaus.pdf}
  \caption{Plot der Abhängigkeit der Ausgangsspannung von der Reflektorspannung.}
  \label{fig:Uausbild}
\end{figure}
\newpage
\textbf{Korrektur:}
Die Reflektorspannung eines Reflexklystrons kann mittels
\begin{align}
  V_\text{R} &= \num{6.74e-6}\cdot f\cdot L\cdot\frac{\sqrt{V_0}}{N}-V_0
  \intertext{bestimmt werden. Da $V_\text{R}$ eingestellt wird, kann auch die Länge
    des Reflexklystrons bestimmt werden. Die verwendeten Werte sind}
    V_\text{R} &= \SI{160}{\volt} \\
    f &= \SI{9}{\giga\hertz} \\
    V_0 &= \SI{300}{\volt} \\
    N &= \num{2.75}\,.
    \intertext{Es folgt}
    L &= \frac{\SI{300}{\volt}+\SI{160}{\volt}}
    {\num{6.74e-6}\sqrt{\si{\kilo\gram\per\ampere\per\second}}
    \cdot\SI{9}{\giga\hertz}\cdot\sqrt{\SI{300}{\volt}}}\cdot\num{2.75} \\
    &= \SI{1.204}{\milli\meter}\,.
\end{align}
\FloatBarrier

\subsection{Bestimmung der Wellenlänge und der Frequenz}
Von den Messwerten in Tabelle \ref{tab:messung2} werden zuerst die Abstände bestimmt.
\begin{table}
    \centering
    \caption{Messwerte der zweiten Messung.}
    \label{tab:messung2}
    \sisetup{table-format=3.1}
    \begin{tabular}{S S}
        \toprule
        {Kurzschluss $\SI{1}{\milli\meter}$} & {Kurzschluss $\SI{3}{\milli\meter}$} \\
        {Position / $\si{\milli\meter}$} & {Position / $\si{\milli\meter}$} \\
        \midrule
         46.0 &  45.0 \\
         71.5 &  69.3 \\
         95.5 &  93.0 \\
        120.5 & 118.5 \\
        \bottomrule
    \end{tabular}
\end{table}
Diese entsprechen der halben Wellenlänge.
Mit dem Mittelwert nach
\begin{align}
    \overline{x} &= \frac{1}{N} \sum_{i=0}^{N} x_i \label{eqn:mittelwert} \\
    \intertext{und dem Mittelwertsfehler}
    \increment\overline{x} &= \sqrt{\frac{1}{N(N-1)}
    \sum_{k=0}^{N}\left( x_k - \overline{x} \right)^2} \label{eqn:mittelwertfehler}
    \intertext{ergeben sich die Wellenlängen}
    λ_1 &= \SI{49.7(9)}{\milli\meter} \\
    λ_3 &= \SI{49(1)}{\milli\meter}\,.
    \intertext{Die Cut-off Wellenlänge beträgt nach Gleichung \eqref{eqn:cutoff}}
    λ_\text{c} &= 2\cdot\SI{22.8}{\milli\meter} = \SI{45.6}{\milli\meter}\,.
    \intertext{Die entsprechenden Frequenzen sind mit}
    f &= \symup{c}\sqrt{\frac{1}{λ_\text{g}^2}+\frac{1}{λ_\text{c}^2}}
    \intertext{und dem Fehler nach der Gaußschen Fehlerfortpflanzung}
    \increment f &= \sqrt{\sum_{j=0}^K \left( \frac{\symup{d}f}{\symup{d}y_j}
	   \increment y_j\right)^{\!\! 2}}
	    \label{eqn:fehler}
    \shortintertext{hier}
    \increment f &= \frac{\symup{c}}
      {λ_\text{g}^3\sqrt{\frac{1}{λ_\text{g}^2}+\frac{1}{λ_\text{c}^2}}}
      \incrementλ \\
    f_1 &= \SI{8.93(7)}{\giga\hertz} \\
    f_3 &= \SI{8.98(9)}{\giga\hertz}\,.
    \intertext{Die Phasengeschwindigkeit folgt mit}
    v_\text{ph} &= λ_\text{g}\cdot f
    \shortintertext{und}
    \increment v_\text{ph} &= \sqrt{(f\incrementλ_\text{g})^2+(λ_\text{g}\increment f)^2}
    \shortintertext{zu}
    v_\text{ph,1} &= (\num{1.479(14)})\symup{c} \\
    v_\text{ph,3} &= (\num{1.468(17)})\symup{c}\,.
\end{align}
\FloatBarrier

\subsection{Bestimmung der Dämpfung des Mikrowellenfeldes}
In Abbildung \ref{fig:daempfbild} sind die Messwerte, sowie die Theoriewerte,
die der auf dem Bauteil aufgedruckten Abbildung entnommen worden sind.
Die Messwerte sind so gewählt, dass bei $d= 0$ keine Dämpfung vorliegt.
Auf der X-Achse ist die Stellung der Dämpfungsschraube gegen das Leistungsverhältnis
auf der Y-Achse aufgetragen.
Als Fitfunktion der Messwerte wird
\begin{align}
    f(d) &= A\cdot\exp(B\cdot d)+C
    \intertext{gewählt. Es folgen die Fitparameter}
    \input{build/messung2-a.tex} \\
    \input{build/messung2-b.tex} \\
    \input{build/messung2-c.tex} \,.
\end{align}

\begin{figure}
  \centering
  \includegraphics[width=0.8\textwidth]{build/messung3_daempf.pdf}
  \caption{Plot der Theoriewerte und Messwerte mit Fitfunktion.}
  \label{fig:daempfbild}
\end{figure}

\FloatBarrier

\subsection{Die SWR Methoden}
\subsubsection{Die direkte Methode}
Bei der direkten Methode wurden, gemäß\, der Anleitung, direkt das SWR
gemessen. Die aufgenommen Messwerte sind in Tabelle \ref{tab:swr_direkt}
zusehen, mit dem dazugehörigen Plot in Abbildung \ref{fig:swr_direkt_plot}.
Die Messwerte wurden mit dem SWR-Meter direkt aufgenommen.
\begin{figure}
  \centering
  \includegraphics[width=0.8\textwidth]{build/messung4_direkt.pdf}
  \caption{Messwerte der direkten Methode.}
  \label{fig:swr_direkt_plot}
\end{figure}

\begin{table}
  \centering
  \caption{Messwerte der direkten Methode.}
  \label{tab:swr_direkt}
  \sisetup{table-format=1.2}
  \begin{tabular}{S[table-format=1.1]SS}
  \toprule
    {Stifttiefe\;/\;$\si{\milli\metre}$} & \text{SWR} & {Korrekturdaten}\\
  \midrule
  0   & 1.35 & 1.0  \\
  1.5 & 1.35 &      \\
  3   & 1.32 & 1.02 \\
  5   & 1.22 & 1.22 \\
  7   & 1.05 & 2.4  \\
  9   &      & 5.0  \\
  \bottomrule
  \end{tabular}
\end{table}

\FloatBarrier

\subsubsection{Die 3-dB-Methode}
Hier wurde die 3dB Methode nach der Anleitung verwendet.
Aufgenommen wurden die Stellen, an denen die Leistung doppelt so groß\, ist wie
im Minimum. Mit Formel \eqref{eqn:3dBM} wird das SWR berechnet zu
\begin{align}
  \text{SWR}_{3dB, \lambda1} &= \num{8.40(15)} \\
  \text{SWR}_{3dB, \lambda3} &= \num{8.29(18)}\,.
\end{align}

\subsubsection{Die Abschwächer Methode}
Bei der Abschwächer Methode wurde am Maximum eine Abschwächung von
$\SI{46}{\decibel}$ und am Minimum eine Abschwächung von
$\SI{16}{\decibel}$ eingestellt um zu erreichen, dass die Leistungen
übereinstimmen.
Mit Formel \eqref{eqn:abschw} folgt ein SWR von
\begin{equation}
  \text{SWR} = \num{31.623}\,.
\end{equation} \\
\textbf{Korrektur:}
\begin{align}
  A_1 &= \SI{20}{\deci\bel} \\
  A_2 &= \SI{37}{\deci\bel} \\
  \text{SWR} &= \num{7.08}
\end{align}
