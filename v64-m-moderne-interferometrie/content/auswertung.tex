\section{Auswertung}
\label{sec:Auswertung}

\subsection{Kontrast}
Die Messwerte in Tabelle~\ref{tab:m1} sind mit einem Wurzelfehler versehen worden,
da eine Zählung immer nach der \textsc{Poisson}-Statistik verteilt ist.
Der Kontrast wird mit Gleichung~\eqref{eqn:kontrast} bestimmt.
Graphisch ist der Verlauf in Abbildung~\ref{fig:m1} zu sehen.
Das Maximum liegt somit bei
\begin{align}
  φ &= \SI{45}{\degree} \\
  \text{Kontrast} &= 0,928\,.
\end{align}

\begin{figure}
  \centering
  \includegraphics{build/p-m1.pdf}
  \caption{Darstellung des Kontrasts als Polarplot, der Winkel ist der
    am Polarisator eingestelle Winkel.}
  \label{fig:m1}
\end{figure}

\begin{table}
  \centering
  \caption{Daten der Messung~1 und der zugehörige Kontrast.}
  \label{tab:m1}
  \input{build/p-m1.tex}
\end{table}

\FloatBarrier
\subsection{Brechungsindex von Glas}
Von den Messwerten in Tabelle~\ref{tab:m3} wird der Mittelwert zu
\begin{align}
  \input{build/p-m3-werte.tex}
  \intertext{bestimmt. Die Phasenverschiebung folgt mit}
  \overline{\Del{Φ}} &= 2 \mpi \cdot \overline{M}
  \shortintertext{zu}
  \input{build/p-m3-phase.tex}\,.
  \intertext{Mit Gleichung~\eqref{eqn:brechFestk} wird der Brechungsindex
    mit den Winkeln}
  \Del{θ} = θ &= \SI{9}{\degree} \wedge \SI{10}{\degree}
  \shortintertext{bestimmt:}
  \input{build/p-m3-n9.tex} \\
  \input{build/p-m3-n10.tex}\,.
\end{align}

\begin{table}
  \centering
  \caption{Messwerte der Messung~3.}
  \label{tab:m3}
  \sisetup{table-format=2.0}
  \begin{tabular}{c S S S S S S S S S S}
    \toprule
    {Messung}         &  1 &  2 &  3 &  4 &  5 &  6 &  7 &  8 &  9 & 10 \\
    {\# Counts ($M$)} & 37 & 36 & 36 & 34 & 35 & 34 & 38 & 35 & 36 & 35 \\
    \midrule
    {Messung}         & 11 & 12 & 13 & 14 & 15 & 16 & 17 & 18 & 19 & 20 \\
    {\# Counts ($M$)} & 35 & 35 & 35 & 33 & 37 & 34 & 35 & 34 & 38 & 34 \\
    \bottomrule
  \end{tabular}
\end{table}

\FloatBarrier
\subsection{Brechungsindex von Luft}
\label{sec:m4}
Die Werte der Messung~4 stehen in Tabelle~\ref{tab:m4-werte}
im Kapitel~\ref{sec:anhang}.
Der Brechungsindex des Behälterinhalts wird mit
\begin{align}
  n_B &= \frac{λ \cdot M}{L} + n_A
  \intertext{berechnet, wobei}
  n_A &= \num{1}
\end{align}
der Brechungsindex des Vakuums ist.
Die so bestimmten Werte stehen in Tabelle~\ref{tab:m4-behbrech}.
Diese werden dann an eine Gerade der Form
\begin{equation}
  n^2 = A \cdot p + B
\end{equation}
mit \texttt{scipy.optimize} gefittet.
Die Fitparameter stehen in Tabelle~\ref{tab:m4-fit}.
Es ergeben sich die gemittelten Werte
\begin{align}
  \input{build/p-m4-amean.tex} \\
  \input{build/p-m4-bmean.tex}\,.
\end{align}
Mit diesen Fitparametern und den vorgegebenen Werten
\begin{align}
  T &= \SI{15}{\celsius} = \SI{288.15}{\kelvin} \\
  p &= \SI{1013}{\milli\bar}
  \intertext{ergibt sich der Brechungsindex zu}
  \input{build/p-m4-normbrech.tex}\,.
\end{align}

\begin{table}
  \centering
  \caption{Fitparameter der Messung~4.}
  \label{tab:m4-fit}
  \input{build/p-m4-fit.tex}
\end{table}
