\section{Diskussion}
\label{sec:Diskussion}
%Die Lage des Kontrastmaximums passt mit der Vorhersage sehr gut überein.
%\rot{Theorieabschnitt}

Die Bestimmung des kompletten Drehwinkels $θ$ in der dritten Messung
ist schwierig, da die Skala sehr klein ist, aber nicht fein.
Deswegen haben wir in die Brechungsindices des Glases für beide Winkel bestimmt.
Je nach Zusammensetzung des Glases liegt der Brechungsindex
zwischen $\num{1.571}$ und $\num{1.68}$ \cite[197]{hecht}.
Dies lässt darauf schließen, dass der tatsächliche Winkel näher an
$\SI{9}{\degree}$ als an $\SI{10}{\degree}$ liegt.

Der Literaturwert des Brechungsindices von Luft ist \cite{brechluft}
\begin{align}
  n_{\text{Luft,Lit}} &= \num{1.000276145(23)}\,.
  \intertext{Damit weicht er stark von dem uns bestimmten Wert}
  \input{build/p-m4-normbrech.tex}
\end{align}
ab. Das kann an dem für die zum Versuchszeitpunkt geltenden Verhältnisse
verwendeten Brechungsindex $n_A$ liegen, oder an der Näherung des
\textsc{Lorenz-Lorentzschen}-Gesetzes.
Eine weitere Fehlerquelle ist, dass im Fit nicht $B = 1$ gesetzt wurde,
wird dies getan, verschlechtert sich das Ergebnis auf $n_0 \approx \num{1.0000003529(15)}$.

In der Messung~4 sollte eigentlich eine Haube aus Plexiglas über das System
gestellt werden, um äußere Einflüße abzuschirmen.
Wenn die Haube jedoch auf die Grundplatte gesetzt wurde, hat sich das System
so dejustiert, dass keine Messung möglich war.
