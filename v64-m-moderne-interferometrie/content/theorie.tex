\section{Zielsetzung}

In diesem Versuch sollen die grundlegenden
Eigenschaften eines Sagnac-Interferometers untersucht,
sowie die Brechungsindizes von Gasen und Festkörpern
bestimmt werden. Das Sagnac-Interferometer bietet hier
gegenüber anderen Interferometern den Vorteil, dass
es stabiler gegenüber äußeren Einflüssen ist.
Dies liegt daran, dass die Strahlen im Interferometer,
nicht wie bei anderen Interferometern, räumlich getrennt laufen,
sondern nahezu die gleiche Wegstrecke zurücklegen.

\section{Theoretische Grundlagen}
\label{sec:theorie}

%\begin{itemize}
%    \item polarisation
%    \item Licht: wellenbeschreibung, überlagerung, interferenz
%    \item brechungsindex von gasen und festkörpern
%    \item kontrast
%\end{itemize}

\subsection{Allgemeine Eigenschaften von Licht}
Licht wird in diesem Versuch als eine elektromagetische Welle angenommen.
Zur Beschreibung genügt die Betrachtung des elektrischen Feldes.
Eine Welle kann definiert werden als
\begin{equation}
  \bfE = \bfE_{0} \exp\!\left[\I \l(\symbf{k} \symbf{r} - ω t\r)\right]\,.
\end{equation}
Hierbei ist $\bfE_{0}$ die Polarisation des Feldes.
Eine Welle kann in zwei Teilwellen zerlegt werden, beide
Schwingungsrichtungen stehen orthogonal auf der Ausbreitungsrichtung.
Die Polarisationsrichtung von Wellen kann mit Polarisationsfiltern
verändert werden.
Solche Filter absorbieren den Teil der Welle, der um $\SI{90}{\degree}$
gegenüber der Filterachse gedreht ist. Es wird der transmittierte
Anteil auf die Filterachse projiziert.

In diesem Versuch werden PBSCs\footnote{polarizing beam splitter Cube}
verwendet. Diese teilen den eintreffenden Strahl in zwei Teilstrahlen auf,
welche orthogonal zueinander den Strahlteiler verlassen.
Außerdem haben die Teilstrahlen eine um $\SI{90}{\degree}$ zueinander
gedrehte Polarisationsrichtung.

\subsection{Polarisation}
\subsubsection{Lineare Polarisation}
Bei linearer Polarisation ist die Orientierung des Feldes konstant,
jedoch ändert die Amplitude Vorzeichen und Betrag.
Für die weitere Beschreibung nehmen wir die beiden Wellen an
\begin{align}
  \bfE_x\l(z, t\r) &= \hat{\symbf{I}}\bfE_{0x} \cos\!\left(kz- ωt\right) \\
  \bfE_y\l(z, t\r) &= \hat{\symbf{J}}\bfE_{0y} \cos\!\left(kz - ωt + \Del{Φ}\right)\,.
\end{align}

Diese beiden Wellen breiten sich in die gleiche Richtung aus, schwingen aber
in unterschiedlichen Ebenen. Die relative Phase zwischen den Wellen wird mit
$\Del{Φ}$ bezeichnet.
Für ein allgemeines $\Del{Φ}$ ist die überlagerte Welle die vektorielle
Summe der beiden Einzelwellen.

\begin{equation}
  \bfE\l(z, t\r) = \bfE_x\l(z, t\r) + \bfE_y\l(z, t\r)
  \label{eqn:ueberlagerung1}
\end{equation}

Wenn $\Del{Φ} = 0$ oder ein ganzzahliges Vielfaches von $\pm 2 \mpi$ ist,
vereinfacht sich Gleichung \eqref{eqn:ueberlagerung1} zu

\begin{equation}
  \bfE\l(z, t\r) = \left(\hat{\symbf{I}}\bfE_{0x} + \hat{\symbf{J}}\bfE_{0y} \right)
  \cos\l(kz - ωt\r)
  \label{eqnb:ueberlagerung2}
\end{equation}

Die resultierende Welle ist wieder linear polarisiert und hat eine feste
Amplitude $\left(\hat{\symbf{I}}\bfE_{0x} + \hat{\symbf{J}}\bfE_{0y} \right)$.

\subsubsection{Zirkulare Polarisation}
Für den Fall, dass eine Welle der anderen um $\Del{Φ} = -\frac{\mpi}{2} + 2 \mpi k$
vorausgeht, mit $k = 0, \pm 1, \pm 2, ...$, und die Amplituden identisch sind
$\symup{E}_{0x} = \symup{E}_{0y} = \symup{E}_{0}$, ergibt sich
für die resultierende Welle
\begin{equation}
  \bfE\l(z, t\r) = \symup{E}_0 \left[\cos\l(kz - ωt\r) +
  \sin\l(kz - ωt\r)\right]\,.
\end{equation}

Die Polarisation von $\bfE$ ändert mit der Zeit seine Richtung,
hier im Uhrzeigersinn. Diese Art von Welle wird rechtszirkular polarisierte
genannt.

\subsection{Interferenz}
Durch die überlagerung von zwei Wellen kann es zu Interferenzeffekten kommen.
Nimmt man zwei Wellen $\bfE_{1}$ und $\bfE_{2}$ mit
\begin{align}
  \bfE_1 &= \bfE_{01} \exp\!\left[\I \l(\symbf{k}_1 \symbf{r} - ωt\r) \right] \\
  \bfE_2 &= \bfE_{02} \exp\!\left[\I \l(\symbf{k}_2 \symbf{r} - ωt + \Del{Φ}\r) \right]
\end{align}
an, ergibt sich die Observable, die Intensität, zu
\begin{align}
  \symup{I} &\propto \left|\bfE_{1} + \bfE_{1}\right|^2 \\
            &= \bfE_{01}^2 + \bfE_{02}^2 + 2\bfE_{01}
            \bfE_{02}\cos\l(δ\r) \cos\l(\Del{Φ}\r)\,.
  \label{eqn:interferenz}
\end{align}

Hierbei ist $δ$ der Polarisationswinkel und $\Del{Φ}$ der
Phasenunterschied zwischen $\bfE_1$ und $\bfE_2$.
Der letzte Term in Gleichung \eqref{eqn:interferenz} wird Interferenzterm
genannt.

Die Art der Interferenz ist demnach abhängig vom Phasenunterschied.
Es gilt
\begin{align}
  \Del{Φ} &= 0, 2\mpi, 4\mpi, \dotsc & &\text{kontstruktive Interferenz} \\
  \Del{Φ} &= \mpi, 3\mpi, 5\mpi, \dotsc & &\text{destruktive Interferenz}\,.
\end{align}

Außerdem ist zu beachten, dass senkrecht zueinander polarisierte
Wellen nicht interferenzfähig sind, was an dem Faktor
$\cos\!\left(\sfrac{\mpi}{2}\right)$ liegt.

\subsection{Brechungsindex bei Gasen}
Propagiert ein Lichtstrahl von Medium A durch eine Gaszelle der Länge $L$ mit
Medium B, so führt das zu einer Phasendifferenz $\Del{Φ}$ relativ gesehen
zu einem Lichtstrahl, der sich nur durch Medium A bewegt. Erzeugt wird diese
Phasendifferenz durch die Änderung der Phasengeschwindigkeit im Medium B
$v_{\text{ph}} = \frac{\symup{c}}{n} = \frac{ω}{k}$,
wobei c die Vakuumlichtgeschwindigkeit darstellt.
Eine Änderung der Phasengeschwindigkeit verursacht eine Änderung der
Wellenzahl $k$ und somit eine Phasenverschiebung gemäß

\begin{align}
  \Del{Φ} &= kL \\
             &= \frac{2\mpi}{λ_{\text{vac}}}\left(n_{B} - n_{A}\right) L\,.
\end{align}

Die Brechungsindizes der Medien A und B sind als $n_{A,B}$ gekennzeichnet und
$λ_{\text{vac}}$ ist die Vakuumwellenlänge.
Im allgemeinen ist der Brechungsindex druckabhängig. Diese Abhängigkeit
wird durch das Lorentz-Lorenz'sche Gesetz definiert gemäß
\begin{equation}
  n \approx \sqrt{1 + \frac{3Ap}{RT}}\,.
  \label{eqn:lorlor}
\end{equation}

Der molare Brechungsindex ist hier $A$, die Gaskonstante $R$ und die Temperatur $T$.
In Interferometern können mit Hilfe von Gaszellen durch die
Druckabhängigkeit Interferenzphänomene analysiert werden.
Zählt man bei einer gegebenen Druckdifferenz die Interferenzmaxima kann der
Brechungsindex bestimmt werden.

\begin{equation}
  M = \frac{\Del{Φ}}{2\mpi} = \frac{n_{B} - n_{A}}{λ_{\text{vac}}} L
\end{equation}

\subsection{Brechungsindex bei Festkörpern}
Bei der Brechung an Festkörpern ist die Druckabhängigkeit verschwindend
gering und wird daher nicht weiter betrachtet. Ein Lichtstrahl wird während
der Propagation gemäß des Snellius'schen Brechungsgesetzes gebrochen. Der
dadurch entstehende Gangunterschied resultiert in einer Phasendifferenz
über den Zusammenhang
\begin{align}
  \Del{Φ} &= \frac{2\mpi T}{\lambda_{vac}}\left( \frac{n -
  \cos\l(θ - θ'\r)}{\cos\l(θ'\r)} - n + 1\right)\,. \\
  \intertext{Die Annahme einer Kleinwinkelnäherung liefert dann}
  \Del{Φ} &= \frac{2\mpi T}{\lambda_{\text{\text{vac}}}}\left( \frac{n - 1}{2n} θ^2 + \symcal{0}(θ^4)\right)\,.
  \label{eqn:deltheta}
\end{align}

Daraus lässt sich nun die Anzahl $M$ an Interferenzmaxima bestimmen
\begin{align}
  M &= \frac{\Del{Φ}}{2\mpi} \approx \frac{T}{λ_{\text{vac}}}
  \frac{n - 1}{2n}\theta^{2}\,.
  \label{eqn:intmaxFestk}
  \intertext{Die Taylorentwicklung liefert im linearen Term}
  M &\approx \frac{T}{λ_{\text{vac}}}\frac{n - 1}{n}\l(2 θ_0 \Del{θ}\r)\,.
  \intertext{Formt man nach dem Brechungsindex $n$ um, folgt}
  n &\approx \left[1 - \frac{M λ_{\text{vac}}}{2 T θ_0 \Del{θ}}\right]^{\!-1}
  \label{eqn:brechFestk}
\end{align}


\subsection{Kontrast}
Der Kontrast eines Interferometers beschreibt die Güte des Auflösungsvermögens
\begin{equation}
  K = \frac{I_{\text{max}} - I_{\text{min}}}{I_{\text{max}} + I_{\text{min}}}
    = \abs{\sin\l(2 θ\r)}\,.
  \label{eqn:kontrast}
\end{equation}

Dabei sind $I_{\text{max}}$ und $I_{\text{min}}$ die maximale und minimale Intensität.
