\section{Auswertung}
\label{sec:Auswertung}
\subsection{Darstellung der Ergebnisse}
In Abbildung~\ref{fig:temperatur} ist der Verlauf der Temperatur
während der beiden Messungen aufgetragen, in der Abbildung~\ref{fig:strom}
der Depolarisationsstrom während der Messungen.

\begin{figure}
  \centering
  \includegraphics[width=0.8\textwidth]{build/p-temperatur.pdf}
  \caption{Temperaturverlauf.}
  \label{fig:temperatur}
\end{figure}
\begin{figure}
  \centering
  \includegraphics[width=0.8\textwidth]{build/p-strom.pdf}
  \caption{Stromverlauf.}
  \label{fig:strom}
\end{figure}
In den beiden Plots sind einmal die Messwerte der Probentemperatur
und des Depolarisationsstromes in Abhängigkeit der Zeit aufgetragen.
\FloatBarrier
Interessanter ist jedoch der Verlauf des Depolarisationsstromes in Abhängigkeit
der Probentemperatur. Dies ist in Abbildung~\ref{fig:korrelation} dargestellt.

\begin{figure}
  \centering
  \includegraphics[width=0.8\textwidth]{build/p-korrelation.pdf}
  \caption{Aufgetragen ist der Depolarisationsstrom in Abhängigkeit der Probentemperatur.}
  \label{fig:korrelation}
\end{figure}

An diese Kurven wird jetzt mit \texttt{SciPy} eine Exponentialfunktion
\begin{equation}
  f\l(x\r) = a \cdot \me^{x \cdot b}
\end{equation}
an die ansteigende Flanke des zweiten Maximums gefittet.

Die Ergebnisse für die Messung~1 sind \\
\begin{align}
  \input{build/p-korrektur-1-a.tex} \\
  \input{build/p-korrektur-1-b.tex}\,.
\end{align}
\begin{figure}
  \centering
  \includegraphics[width=0.8\textwidth]{build/p-korrektur-1.pdf}
  \caption{Exponentialfit zur Korrektur der Messwerte der Messung 1.}
  \label{fig:korrektur-1}
\end{figure}

Die Ergebnisse für die Messung~2 sind \\
\begin{align}
  \input{build/p-korrektur-2-a.tex} \\
  \input{build/p-korrektur-2-b.tex}\,.
\end{align}
\begin{figure}
  \centering
  \includegraphics[width=0.8\textwidth]{build/p-korrektur-2.pdf}
  \caption{Exponentialfit zur Korrektur der Messwerte der Messung 2.}
  \label{fig:korrektur-2}
\end{figure}
\FloatBarrier
\subsection{Bestimmmung der Aktivierungsenergie - Methode I}
Mit einer linearen Ausgleichsgeraden der Form
\begin{align}
  f\l(x\r) &= m \cdot x + b\,, \label{eqn:gerade}
  \intertext{wobei $x = t$ und $f\l(x\r) = T$, erhalten wir die Heizraten}
  \input{build/p-rate-m-1.tex} \\
  \input{build/p-rate-m-2.tex} \\
  \input{build/p-rate-b-1.tex} \\
  \input{build/p-rate-b-2.tex}\,.
\end{align}

\begin{figure}
  \centering
  \includegraphics[width=0.8\textwidth]{build/p-aktivierung-1-heiz-1.pdf}
  \caption{Bestimmung der Heizrate für die Messung 1 mittels einer Ausgleichsgeraden.}
  \label{fig:akt-1-heiz-1}
\end{figure}

\begin{figure}
  \centering
  \includegraphics[width=0.8\textwidth]{build/p-aktivierung-1-heiz-2.pdf}
  \caption{Bestimmung der Heizrate für die Messung 2 mittels einer Ausgleichsgeraden.}
  \label{fig:akt-1-heiz-2}
\end{figure}
\FloatBarrier
In den Abbildungen~\ref{fig:akt-1-methode-1} und \ref{fig:akt-2-methode-1}
ist je der Logarithmus der korrigierten Depolarisationsstromwerte gegen die inverse Temperatur aufgetragen.
An die orangenen Messwerte wird eine Ausgleichsgerade~\eqref{eqn:gerade} gelegt.
Diese hat die jeweiligen Parameter
\begin{align}
  \input{build/p-aktivierung-1-methode-1-m.tex} \\
  \input{build/p-aktivierung-1-methode-1-b.tex} \\
  \input{build/p-aktivierung-2-methode-1-m.tex} \\
  \input{build/p-aktivierung-2-methode-1-b.tex}\,.
\end{align}
Es ergibt sich mit
\begin{align}
  W &= -k_B \cdot m \\
  \input{build/p-aktivierung-1-methode-1-W.tex}
    \input{build/p-aktivierung-1-methode-1-W-eV.tex} \\
  \input{build/p-aktivierung-2-methode-1-W.tex}
    \input{build/p-aktivierung-2-methode-1-W-eV.tex}\,.
\end{align}

\begin{figure}
  \centering
  \includegraphics[width=0.8\textwidth]{build/p-aktivierung-1-methode-1.pdf}
  \caption{Aufgetragen ist der Logarithmus des Depolarisationsstroms gegen die
    inverse Temperatur für die erste Messreihe.
    Einige der korrigierten Depolarisationsstromwerte sind negativ,
    sodass kein Logarithmus gebildet werden kann.}
  \label{fig:akt-1-methode-1}
\end{figure}

\begin{figure}
  \centering
  \includegraphics[width=0.8\textwidth]{build/p-aktivierung-2-methode-1.pdf}
  \caption{Aufgetragen ist der Logarithmus des Depolarisationsstroms gegen die
    inverse Temperatur für die zweite Messreihe.
    Einige der korrigierten Depolarisationsstromwerte sind negativ,
    sodass kein Logarithmus gebildet werden kann.}
  \label{fig:akt-2-methode-1}
\end{figure}

\newpage

\subsection{Bestimmmung der Aktivierungsenergie - Methode II}
Für die zweite Methode schauen wir uns erst die korrigierte Stromkurve an.
Diese ist für Messung~1 in Abbildung~\ref{fig:stromr1}, für Messung~2 in
Abbildung~\ref{fig:stromr2} zu sehen. In den weiteren Schritten werden
bezüglich der Messung~1 die positiven Messwerte zwischen
$T \approx \SI{240}{\kelvin}$ und $\SI{270}{\kelvin}$ betrachtet,
für die zweite Messung zwischen $T \approx \SI{235}{\kelvin}$ und
$\SI{270}{\kelvin}$.

\begin{figure}
  \centering
  \includegraphics[width=0.8\textwidth]{build/p-aktivierung-1-stromr-1.pdf}
  \caption{Die korrigierte Stromkurve der Messung~1.}
  \label{fig:stromr1}
\end{figure}
\begin{figure}
  \centering
  \includegraphics[width=0.8\textwidth]{build/p-aktivierung-2-stromr-1.pdf}
  \caption{Die korrigierte Stromkurve der Messung~2.}
  \label{fig:stromr2}
\end{figure}

Es wird dann das Integral
\begin{align}
  &\int_T^{T'} \!\! i\l(T\r) \, \dif{T'}
  \intertext{mit der Trapezregel}
  I &= \frac{b - a}{2} \left(i\l(a\r) + i\l(b\r)\right)
  \intertext{bestimmt, mit einer in \texttt{python} selbstgeschriebenen Methode.
    So, dass dann}
  &\ln\left(\frac{\int_T^{T'} i(T) \symup{d}T'}{i(T) b}\right)
\end{align}
gegen die inverse Temperatur in den Abbildungen~\ref{fig:fit1} und
\ref{fig:fit2} aufgetragen werden kann.
Eine lineare Ausgleichsrechnung ergibt für die Gerade
\begin{align}
  y &= m \cdot x + b
  \intertext{und die Umformungen}
  W &= k_B \cdot m \\
  τ_0 &= \me^{-b}
  \intertext{die Werte}
  \input{build/p-akt-1-meth-2-m.tex} \\
  \input{build/p-aktivierung-1-methode-2-W.tex}
    \input{build/p-aktivierung-1-methode-2-W-eV.tex} \\
  \input{build/p-akt-1-meth-2-b.tex} \\
  \input{build/p-akt-1-meth-2-tau.tex} \\
  \input{build/p-akt-2-meth-2-m.tex} \\
  \input{build/p-aktivierung-2-methode-2-W.tex}
    \input{build/p-aktivierung-2-methode-2-W-eV.tex} \\
  \input{build/p-akt-2-meth-2-b.tex} \\
  \input{build/p-akt-2-meth-2-tau.tex}
\end{align}

\begin{figure}
  \centering
  \includegraphics[width=0.8\textwidth]{build/p-aktivierung-1-fit-1.pdf}
  \caption{Darstellung der logarithmierten, integrierten Werte in
    Abhängigkeit der inversen Temperatur für die Messung~1.
    Ebenfalls eingezeichnet ist die Ausgleichsgerade.}
    \label{fig:fit1}
\end{figure}

\begin{figure}
  \centering
  \includegraphics[width=0.8\textwidth]{build/p-aktivierung-2-fit-1.pdf}
  \caption{Darstellung der logarithmierten, integrierten Werte in
    Abhängigkeit der inverssen Temperatur für die Messung~2.
    Ebenfalls eingezeichnet ist die Ausgleichsgerade.}
    \label{fig:fit2}
\end{figure}

\FloatBarrier
\subsection{Bestimmung der Relaxationszeit}
Wir bestimmen die Relaxationszeit, indem wir in Gleichung \eqref{eqn:theo-tau}
die folgenden Werte einsetzen:
\begin{align}
  b_1 &= \SI{1.79(2)}{\kelvin\per\minute} \\
  T_{\text{max}, 1} &= \SI{260.35}{\kelvin} \\
  W_{1,1} &= \SI{1.8(1)e-19}{\joule} \\
  τ_{1,1}\l(T_{\text{max},1}\r) &= \boxed{\SI{2.9(2)}{\second}} \\
  τ_{1,1,0} &= \boxed{\SI{5(15)e-22}{\second}} \\
  W_{1,2} &= \SI{1.72(5)e-19}{\joule} \\
  τ_{1,2}\l(T_{\text{max},2}\r) &= \boxed{\SI{3.04(9)}{\second}} \\
  τ_{1,2,0} &= \boxed{\SI{5(7)e-21}{\second}}\,.
\end{align}
Die Abbildungen \ref{fig:relax1} und \ref{fig:relax2} zeigen die damit
bestimmten Relaxationszeiten in Abhängigkeit der Temperatur.

\begin{figure}
  \centering
  \includegraphics[width=0.8\textwidth]{build/p-taukurve1.pdf}
  \caption{Temperaturabhängigkeit der Relaxationszeit mit den Werten der Messung~1.}
  \label{fig:relax1}
\end{figure}

\begin{align}
  b_2 &= \SI{1.106(9)}{\kelvin\per\minute} \\
  T_{\text{max},2} &= \SI{255.55}{\kelvin} \\
  W_{2,1} &= \SI{1.59(8)e-19}{\joule} \\
  τ_{2,1}\l(T_{\text{max},2}\r) &= \boxed{\SI{5.13(26)}{\second}} \\
  τ_{2,1,0} &= \boxed{\SI{1(3)e-19}{\second}} \\
  W_{2,2} &= \SI{1.52(5)e-19}{\joule} \\
  τ_{2,2}\l(T_{\text{max},2}\r) &= \boxed{\SI{5.36(18)e-18}{\second}} \\
  τ_{2,2,0} &= \boxed{\SI{1(2)e-18}{\second}}
\end{align}

\begin{figure}
  \centering
  \includegraphics[width=0.8\textwidth]{build/p-taukurve2.pdf}
  \caption{Temperaturabhängigkeit der Relaxationszeit mit den Werten der Messung~2.}
  \label{fig:relax2}
\end{figure}
