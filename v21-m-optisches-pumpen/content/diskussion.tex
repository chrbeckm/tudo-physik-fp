\section{Diskussion}
\label{sec:Diskussion}

Bei der Messung der Vertikalkomponente des Erdmagnetfeldes gibt es aufgrund der Messmethode
verschiedene Unsicherheiten.
Zum einen ist hier die Ablesegenauigkeit des Einstellungsknopfes zu nennen,
andererseits die Feeinustierung des Knopfes, damit ist gemeint ob bei der Einstellung 0
tatsächlich kein Strom fließt.
Das ablesen am Oszilloskop selber ist hier jedoch die größte Fehlerquelle und
für die Abweichung verantwortlich.

Die Messung der Kernspins in Kapitel \ref{sec:5.2} liefert vom Zahlenwert her
einen Wert nahe den theoretischen Werten, die Unsicherheiten der experimentellen Werte sind jedoch sehr klein.
Da diese Werte aus den Steigungen der Ausgleichsgeraden folgen und hier die Unsicherheiten ebenfalls klein sind, ist auf einen systematischen Fehler zu schließen.
Dieser ist eventuell in den Potentiometern zu suchen.

Das Isotpenverhältnis ist nahe dem das in der Quelle gegeben ist.
Die Abweichungen sind auf die Auflösung auf dem Oszilloskop zurückzuführen.

Das in dem letzten Auswertungsteil bestimmte Verhältnis
liegt nahe dem in der Anleitung gegebenen.
Dieses liegt sogar in der 1-$σ$-Umgebung.
