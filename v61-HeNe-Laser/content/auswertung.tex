\section{Auswertung}
\label{sec:Auswertung}
\subsection{Stabilität}
In Abbildung \ref{fig:stabi-2konkav} sind die Messwerte der Messung aufgetragen,
zusammen mit zwei Theoriekurven nach Gleichung \eqref{eqn:stabi}.
Die Theoriekurve hat die Form
\begin{align}
    \left(g_1g_2\right)_\text{Theorie} &= \num{15000}\cdot
        \left(1-\frac{L}{\SI{1}{\meter}}\right)^{\!2} \label{eqn:g1g2}.
    \intertext{Die Minima liegen bei}
    L_\text{Messwerte} &= \SI{1.02}{\meter} \\
    L_\text{Theorie} &= \SI{1}{\meter}\,.
\end{align}
\begin{figure}
    \centering
    \includegraphics[width=0.8\textwidth]{build/stabi-2konfokal.pdf}
    \caption{Messwerte und Theoriewerte für 2 konfokale Spiegel.}
    \label{fig:stabi-2konkav}
\end{figure}
\FloatBarrier
Für die Kombination aus planarem und konfokalen Spiegel lauten die Theoriegleichungen
\begin{align}
    \left(g_1g_2\right)_\text{rot} &= 200\cdot\left(1-\frac{L}{\SI{1.4}{\meter}}\right) \\
    \left(g_1g_2\right)_\text{schwarz} &= 200\cdot\left(1-\frac{L}{\SI{1}{\meter}}\right) \\
    \left(g_1g_2\right)_\text{grün} &= 500\cdot\left(1-\frac{L}{\SI{0.7}{\meter}}\right)\,.
\end{align}
Es gibt folglich keine Minima.
Die Werte sind in Abbildung \ref{fig:stabi-planar-konkav} dargestellt.
\begin{figure}
    \centering
    \includegraphics[width=0.8\textwidth]{build/stabi-konfokal-planar.pdf}
    \caption{Messwerte und Theoriewerte für je einen planaren und konfokalen Spiegel.}
    \label{fig:stabi-planar-konkav}
\end{figure}
\FloatBarrier

\subsection{Transversal Elektische Moden}
Die Ergebnisse der Messung der $\text{TEM}_{00}$ sind in Abbildung \ref{fig:tem00} zu sehen.
Es wird ein Fit der Form
\begin{align}
    f(x) &= a\cdot \exp\left(-\frac{(x-μ)^2}{2\cdot σ^2}\right)
    \intertext{angesetzt. Es kann keine normierte Gauß-Funktion verwendet,
      da die Intensität ebenfalls nicht normiert ist.
      Es folgen mit scipy \cite{scipy} die Fitparameter}
    \input{build/tem00-mu.tex} \\
    \input{build/tem00-sigma.tex} \\
    \input{build/tem00-a.tex}\,.
\end{align}
\begin{figure}
    \centering
    \includegraphics[width=0.8\textwidth]{build/tem00.pdf}
    \caption{Messwerte für die $\text{TEM}_{00}$.}
    \label{fig:tem00}
\end{figure}
\begin{figure}
  \centering
  \includegraphics[width=0.8\textwidth]{build/tem10.pdf}
  \caption{Messwerte für die $\text{TEM}_{10}$.}
  \label{fig:tem10}
\end{figure}

\FloatBarrier
Für die Messung der $\text{TEM}_{10}$ sind in Abbildung \ref{fig:tem10}
die Messwerte aus Tabelle \ref{tab:tem10-neu} aufgetragen.
Der Fit der Form
\begin{align}
    f(x) &= a\cdot (x-μ_a)^2\exp\left(-\frac{(x-μ_b)^2}{b^2}\right)
    \intertext{bringt die Paramter}
    \input{build/tem10-mua.tex} \\
    \input{build/tem10-mub.tex} \\
    \input{build/tem10-a.tex} \\
    \input{build/tem10-b.tex}\,.
\end{align}

\FloatBarrier

\subsection{Polarisation}
Die Polarisationsmesswerte sind in Abbildung \ref{fig:polarisation} als Polarplot abgebildet.
Als Radius ist die Intensität in $\si{\micro\ampere}$ gesetzt.
Der Winkel entspricht dem Winkel des Polarisators.
\begin{figure}
    \centering
    \includegraphics[width=0.6\textwidth]{build/polar.pdf}
    \caption{Polarplot der Polarisationsmessung.}
    \label{fig:polarisation}
\end{figure}
\FloatBarrier

\subsection{Wellenlänge}
Die theoretische Wellenlänge eines Helium-Neon Lasers im sichtbaren Bereich beträgt laut \cite{eichler}
\begin{equation}
    λ_\text{theo} = \SI{633}{\nano\meter}\,.
\end{equation}
Mit der Gleichung \eqref{eqn:gitter} werden die Wellenlängen pro Maximum gebildet, die Werte stehen in
Tabelle \ref{tab:gitter}.
Die Wellenlängen werden dann nach
\begin{align}
    \overline{λ} &= \frac{1}{N} \sum_{i=0}^{N} λ_i \label{eqn:mittelwert}
    \intertext{mit dem Fehler}
    \increment\overline{λ} &= \sqrt{ \frac{1}{N(N-1)} \sum_{k=0}^{N} \left( λ_k - \overline{λ} \right)^2}
    \label{eqn:mittelwertfehler}
\end{align}
gemittelt.
Es folgen
\begin{align}
  \input{build/gitter100.tex} \\
  \input{build/gitter80.tex}\,.
\end{align}

\begin{table}
    \centering
    \caption{Messwerte und Ergebnisse der Wellenlängenbestimmung.}
    \label{tab:gitter}
    \begin{tabular}{S[table-format=1.0] S[table-format=2.1] S[table-format=3.1]
                | S[table-format=1.0] S[table-format=2.1] S[table-format=3.1]}
        \toprule
        \multicolumn{3}{c|}{$100\frac{\text{Linien}}{\si{\milli\meter}}$} &
          \multicolumn{3}{c}{$80\frac{\text{Linien}}{\si{\milli\meter}}$} \\
        {$k$} & {$d_k\;/\;\si{\centi\meter}$} & {$λ\;/\;\si{\nano\meter}$} &
          {$k$} & {$d_k\;/\;\si{\centi\meter}$} & {$λ\;/\;\si{\nano\meter}$} \\
        \midrule
        1 &  8.5 & 633.1 & 1 &  6.8 & 633.5 \\
        1 &  8.5 & 633.1 & 1 &  6.7 & 624.2 \\
        2 & 17.2 & 636.6 & 2 & 13.5 & 626.5 \\
        2 & 17.1 & 632.9 & 2 & 13.5 & 626.5 \\
        3 & 25.9 & 632.6 & 3 & 20.3 & 624.1 \\
        3 & 25.9 & 632.6 & 3 & 20.4 & 627.1 \\
          &      &       & 4 & 27.3 & 624.1 \\
          &      &       & 4 & 27.5 & 627.1 \\
        \bottomrule
    \end{tabular}
\end{table}
