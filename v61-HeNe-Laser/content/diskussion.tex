\section{Diskussion}
\label{sec:Diskussion}
Bei der Stabilitätsmessung mit zwei konfokalen Spiegeln passen die in der Versuchsanleitung und auf den Spiegeln stehenden Werten.
Aus der Abbildung \ref{fig:stabi-2konkav} ist zu erkennen, dass das Minima zu Spiegeln mit einer Brennweite von $\SI{1}{\meter}$ passt.

Die Messung mit einem planaren und einem konfokalen Spiegel ist mit vielen statistischen Fehlern versehen,
da die Intensitäten sehr klein sind und durch den planaren Spiegeln ein Nachjustieren sehr störungsanfällig ist.
Zudem passen hier die Steigungen zwischen Theorie und Messwerten nicht gut überein.
Der konfokale Spiegel müsste eine Brennweite von $\SI{0.7}{\meter}$ haben um zu den Werten zu passen.

Die Messung der $\text{TEM}_{00}$ stimmt sehr gut mit der Theorie überein.
Die $\text{TEM}_{10}$ war schwierig zu erzeugen.
Es ist jedoch in Abbildung \ref{fig:tem10} zu sehen,
dass die gemessene Intensitätsverteilung gut mit der Theorie übereinstimmt.

Der Laserstrahl ist nachweisbar linear polarisiert, da in der Polarisationsmessung eine deutliche Achse zu erkennen ist.

Mit dem feineren Gitter folgt eine Wellenlänge die sehr nahe an der Theoretischen Wellenlänge ist.
Hier sind die Fehlerquellen die Ausrichtung des Schirms und die Messung der Abstände, diese erfolgte mit einem Maßband.
Für das gröbere Gitter folgen schon für die einzelnen Wellenlängen zu kleine Wellenlängen.
Dies deutet auf einen systematischen Fehler hin.
